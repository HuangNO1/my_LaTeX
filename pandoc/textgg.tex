% Options for packages loaded elsewhere
\PassOptionsToPackage{unicode}{hyperref}
\PassOptionsToPackage{hyphens}{url}
%
\documentclass[
]{article}
\usepackage{lmodern}
\usepackage{amssymb,amsmath}
\usepackage{ifxetex,ifluatex}
\ifnum 0\ifxetex 1\fi\ifluatex 1\fi=0 % if pdftex
  \usepackage[T1]{fontenc}
  \usepackage[utf8]{inputenc}
  \usepackage{textcomp} % provide euro and other symbols
\else % if luatex or xetex
  \usepackage{unicode-math}
  \defaultfontfeatures{Scale=MatchLowercase}
  \defaultfontfeatures[\rmfamily]{Ligatures=TeX,Scale=1}
\fi
% Use upquote if available, for straight quotes in verbatim environments
\IfFileExists{upquote.sty}{\usepackage{upquote}}{}
\IfFileExists{microtype.sty}{% use microtype if available
  \usepackage[]{microtype}
  \UseMicrotypeSet[protrusion]{basicmath} % disable protrusion for tt fonts
}{}
\makeatletter
\@ifundefined{KOMAClassName}{% if non-KOMA class
  \IfFileExists{parskip.sty}{%
    \usepackage{parskip}
  }{% else
    \setlength{\parindent}{0pt}
    \setlength{\parskip}{6pt plus 2pt minus 1pt}}
}{% if KOMA class
  \KOMAoptions{parskip=half}}
\makeatother
\usepackage{xcolor}
\IfFileExists{xurl.sty}{\usepackage{xurl}}{} % add URL line breaks if available
\IfFileExists{bookmark.sty}{\usepackage{bookmark}}{\usepackage{hyperref}}
\hypersetup{
  hidelinks,
  pdfcreator={LaTeX via pandoc}}
\urlstyle{same} % disable monospaced font for URLs
\usepackage{longtable,booktabs}
% Correct order of tables after \paragraph or \subparagraph
\usepackage{etoolbox}
\makeatletter
\patchcmd\longtable{\par}{\if@noskipsec\mbox{}\fi\par}{}{}
\makeatother
% Allow footnotes in longtable head/foot
\IfFileExists{footnotehyper.sty}{\usepackage{footnotehyper}}{\usepackage{footnote}}
\makesavenoteenv{longtable}
\usepackage{graphicx}
\makeatletter
\def\maxwidth{\ifdim\Gin@nat@width>\linewidth\linewidth\else\Gin@nat@width\fi}
\def\maxheight{\ifdim\Gin@nat@height>\textheight\textheight\else\Gin@nat@height\fi}
\makeatother
% Scale images if necessary, so that they will not overflow the page
% margins by default, and it is still possible to overwrite the defaults
% using explicit options in \includegraphics[width, height, ...]{}
\setkeys{Gin}{width=\maxwidth,height=\maxheight,keepaspectratio}
% Set default figure placement to htbp
\makeatletter
\def\fps@figure{htbp}
\makeatother
\setlength{\emergencystretch}{3em} % prevent overfull lines
\providecommand{\tightlist}{%
  \setlength{\itemsep}{0pt}\setlength{\parskip}{0pt}}
\setcounter{secnumdepth}{-\maxdimen} % remove section numbering

\author{}
\date{}

\begin{document}

《数据库系统SSD7》

课程实验指导书

谭长庚 编写

课程编号 { 390222Z10}

总 学 时 { 48}

实验学时 { 8}

课外学时 { 20}

{}

中南大学计算机学院

2020年1月

\hypertarget{ux5b9eux9a8c1ux6570ux636eux5e93ux4e0eux8868ux7684ux57faux672cux64cdux4f5c}{%
\section{实验1《数据库与表的基本操作》}\label{ux5b9eux9a8c1ux6570ux636eux5e93ux4e0eux8868ux7684ux57faux672cux64cdux4f5c}}

实验学时: { 2}

每组人数: { 1}

实验类型: { 1} (1:基础性 2:综合性 3:设计性 4:研究性)

实验要求: { 1} (1:必修 2:选修 3:其它)

实验类别: { 3} (1:基础 2:专业基础 3:专业 4:其它)

{}

一、实验目的

\begin{quote}
1.熟练掌握一种DBMS的使用方法,完成数据库的创建、删除和连接;数据表的建立、删除;表结构的修改。

2. 加深对表的实体完整性、参照完整性和用户自定义完整性的理解。
\end{quote}

二、实验内容

\begin{quote}
在以下实验中,使用学生-课程数据库,它描述了学生的基本信息、课程的基本信息及学生选修课程的信息。用SQL语句分别完成。

1.~
创建学生-课程数据库,数据文件名为student\_data、大小10M,日志文件名为student\_log、大小5M的新数据库,该数据库名为student\_xxxxxx,xxxxxx表示班级学号。1801班12号,数据库名为student\_180112。

2.~ 创建学生关系表S :
\end{quote}

\begin{longtable}[]{@{}lllll@{}}
\toprule
\endhead
\begin{minipage}[t]{0.17\columnwidth}\raggedright
\begin{quote}
学号
\end{quote}\strut
\end{minipage} & \begin{minipage}[t]{0.17\columnwidth}\raggedright
\begin{quote}
姓名
\end{quote}\strut
\end{minipage} & \begin{minipage}[t]{0.17\columnwidth}\raggedright
\begin{quote}
性别
\end{quote}\strut
\end{minipage} & \begin{minipage}[t]{0.17\columnwidth}\raggedright
\begin{quote}
年龄
\end{quote}\strut
\end{minipage} & \begin{minipage}[t]{0.17\columnwidth}\raggedright
所在系\strut
\end{minipage}\tabularnewline
\begin{minipage}[t]{0.17\columnwidth}\raggedright
\begin{quote}
Sno
\end{quote}\strut
\end{minipage} & \begin{minipage}[t]{0.17\columnwidth}\raggedright
\begin{quote}
Sname
\end{quote}\strut
\end{minipage} & \begin{minipage}[t]{0.17\columnwidth}\raggedright
\begin{quote}
Ssex
\end{quote}\strut
\end{minipage} & \begin{minipage}[t]{0.17\columnwidth}\raggedright
Sage\strut
\end{minipage} & \begin{minipage}[t]{0.17\columnwidth}\raggedright
sdept\strut
\end{minipage}\tabularnewline
\bottomrule
\end{longtable}

\begin{quote}
(3)~ 创建课程关系表C :
\end{quote}

\begin{longtable}[]{@{}llll@{}}
\toprule
\endhead
\begin{minipage}[t]{0.22\columnwidth}\raggedright
\begin{quote}
课程号
\end{quote}\strut
\end{minipage} & \begin{minipage}[t]{0.22\columnwidth}\raggedright
\begin{quote}
课程名
\end{quote}\strut
\end{minipage} & \begin{minipage}[t]{0.22\columnwidth}\raggedright
\begin{quote}
先行课
\end{quote}\strut
\end{minipage} & \begin{minipage}[t]{0.22\columnwidth}\raggedright
\begin{quote}
学分
\end{quote}\strut
\end{minipage}\tabularnewline
\begin{minipage}[t]{0.22\columnwidth}\raggedright
Cno\strut
\end{minipage} & \begin{minipage}[t]{0.22\columnwidth}\raggedright
\begin{quote}
Cname
\end{quote}\strut
\end{minipage} & \begin{minipage}[t]{0.22\columnwidth}\raggedright
\begin{quote}
Cpno
\end{quote}\strut
\end{minipage} & \begin{minipage}[t]{0.22\columnwidth}\raggedright
\begin{quote}
ccredit
\end{quote}\strut
\end{minipage}\tabularnewline
\bottomrule
\end{longtable}

\begin{quote}
(4)~ 创建学生-课程表SC :
\end{quote}

\begin{longtable}[]{@{}lll@{}}
\toprule
\endhead
\begin{minipage}[t]{0.30\columnwidth}\raggedright
\begin{quote}
学号
\end{quote}\strut
\end{minipage} & \begin{minipage}[t]{0.30\columnwidth}\raggedright
\begin{quote}
课程号
\end{quote}\strut
\end{minipage} & \begin{minipage}[t]{0.30\columnwidth}\raggedright
\begin{quote}
成绩
\end{quote}\strut
\end{minipage}\tabularnewline
\begin{minipage}[t]{0.30\columnwidth}\raggedright
\begin{quote}
Sno
\end{quote}\strut
\end{minipage} & \begin{minipage}[t]{0.30\columnwidth}\raggedright
\begin{quote}
Cno
\end{quote}\strut
\end{minipage} & \begin{minipage}[t]{0.30\columnwidth}\raggedright
\begin{quote}
grade
\end{quote}\strut
\end{minipage}\tabularnewline
\bottomrule
\end{longtable}

\begin{quote}
(5)~ 将以上创建表S、C、SC的SQL命令以
.SQL文件的形式保存在磁盘上。在表中加入至少4个元组,第一个为本人信息。

(6)~ 在表S上增加``出生日期''与``身高''属性列。

(7)~ 删除表S的``身高'' 属性列。

(8)备份数据库,再还原。
\end{quote}

三、实验要求:

1.预习数据库与表的建立方法;

2.预习备份数据库,还原数据库的方法;

3. 预习表结构的修改方法。

四、实验步骤

1.创建数据库或连接已建立的数据库。

2.在当前数据库上建立新表。

3.定义表的结构。

4. 修改表的结构。

五、实验报告

1.完成本项目实验后,学生应完成实验报告。

2.实验报告格式与要求见附录1。

\hypertarget{ux5b9eux9a8c2ux6570ux636eux8868ux67e5ux8be2ux4e0eux66f4ux65b0}{%
\section{实验2《数据表查询与更新》}\label{ux5b9eux9a8c2ux6570ux636eux8868ux67e5ux8be2ux4e0eux66f4ux65b0}}

实验学时: { 4}

每组人数: { 1}

实验类型: { 2} (1:基础性 2:综合性 3:设计性 4:研究性)

实验要求: { 1} (1:必修 2:选修 3:其它)

实验类别: { 3} (1:基础 2:专业基础 3:专业 4:其它)

{}

一、实验目的

1.
熟悉和掌握对数据表中数据的查询操作和SQL命令的使用,学会灵活熟练的使用SQL
语句的各种形式,加深理解关系运算的各种操作(尤其是关系的选择,投影,连接和除运算);

2.
熟悉和掌握数据表中数据的插入、修改、删除操作和命令的使用(熟悉使用UPDATE/INSERT/DELETE语句进行表操作);加深理解表的定义对数据更新的作用。

二、实验内容

\begin{quote}
(一)在表S,C,SC上完成以下查询:

1.~查询学生的基本信息;

2.~查询``CS''系学生的基本信息;

3.~查询``CS''系学生年龄不在19到21之间的学生的学号、姓名;

4.~找出``CS''系年龄最大的学生,显示其学号、姓名;

5.~找出各系年龄最大的学生,显示其学号、姓名;

6.~统计``CS''系学生的人数;

7.~统计各系学生的人数,结果按升序排列;

8.~按系统计各系学生的平均年龄,结果按降序排列;

9.~查询无先修课的课程的课程名和学时数;

10.统计每位学生选修课程的门数、学分及其平均成绩;

11.统计选修每门课程的学生人数及各门课程的平均成绩;

12.找出平均成绩在85分以上的学生,结果按系分组,并按平均成绩的升序排列;

13.查询选修了``1''或``2''号课程的学生学号和姓名;

14.查询选修了课程名为``数据库系统''且成绩在60分以下的学生的学号、姓名和成绩;

15.查询每位学生选修了课程的学生信息(显示:学号,姓名,课程号,课程名,成绩);

16.查询没有选修课程的学生的基本信息;

17.查询选修了3门以上课程的学生学号;

18.查询选修课程成绩至少有一门在80分以上的学生学号;

19.查询选修课程成绩均在80分以上的学生学号;

(二)在表S、C、SC中完成下列更新:

1.~ 将数据分别插入表S、C、SC;

2.~ 将表S、C、SC中的数据保存在磁盘上。

3.~
在表S、C、SC上练习数据的插入、修改、删除操作。(比较在表上定义/未定义主码(Primary
Key)或外码(Foreign Key)时的情况)

4.~ 将表S、C、SC中的数据全部删除,再利用磁盘上备份的数据来恢复数据。

5.~
如果要在表SC中插入某个学生的选课信息(如:学号为``2007001005'',课程号为``c123'',成绩待定),应如何进行?

6.~ 求各系学生的平均成绩,并把结果存入数据库;

7.~ 将``CS''系全体学生的成绩置零;

8.~ 删除``CS''系全体学生的选课记录;

9.~ 删除学号为``S1''的相关信息;

10.将学号为``S1''的学生的学号修改为``S001'';

11.把平均成绩大于80分的男同学的学号和平均成绩存入另一个表S------GRADE(SNO,AVG\_GRADE);

12.~把选修了课程名为``数据结构''的学生的成绩提高10\%;

13.~把选修了``C2''号课程,且成绩低于该门课程的平均成绩的学生成绩删除掉。
\end{quote}

三、实验要求:

\begin{quote}
1. 熟悉DBMS 工作环境;

2.连接到学生-课程数据库

3.复习对表中数据查询SQL语言命令;复习对表中数据的插入、修改和删除的SQL语言命令,了解这些更新语句的基本语法和用法。
\end{quote}

四、实验步骤

\begin{quote}
\textbf{查询:}

1.在表S、C、SC上进行简单查询、连接查询、嵌套查询;

2.使用聚合函数的查询、对数据分组查询、对数据的排序查询。
\end{quote}

\textbf{插入:}

\begin{quote}
1.用SQL命令将数据插入当前数据库的表S、C、SC中;

2.用SQL命令形式修改表S、C、SC中的数据;

3.~用SQL命令形式~ 删除表S、C、SC中的数据。
\end{quote}

五、实验报告

1.完成本项目实验后,完成实验报告;

2.实验报告格式与要求见附件。

\textbf{附录 实例(以Sqlsever为例)}

\begin{enumerate}
\def\labelenumi{\arabic{enumi}.}
\item
  对于student表,将所有专业号为`001'的,并且入学年份为2006的学生,或是专业号为`003',并且年龄小于20岁的学生的班级号改为`001'。
\end{enumerate}

\begin{quote}
步骤:新建查询-输入代码:
\end{quote}

use edudb----假设数据库为edudb

go

UPDATE student---student是 edudb中的学生信息表

SET classno='001'

WHERE spno='001' AND entime='2006' or spno='003' and
(2008-birthday)\textless20

\begin{enumerate}
\def\labelenumi{\arabic{enumi}.}
\setcounter{enumi}{1}
\item
  对于student表,删掉所有年龄小于20岁,并且专业号为`003'的学生的记录。
\end{enumerate}

\begin{quote}
步骤:新建查询-输入代码:
\end{quote}

use edudb

go

DELETE FROM student

WHERE (2008-birthday)\textless20 and spno='003'

\begin{enumerate}
\def\labelenumi{\arabic{enumi}.}
\setcounter{enumi}{2}
\item
  对于student表,插入一条新记录,它的具体信息为,学号:20180302、姓名:李伟龙、性别:男、出生日期:19880808、院系编号:`001'、专业编号:
\end{enumerate}

\begin{quote}
`01'、班级号:`003'、入学时间:20070901。

步骤:新建查询-输入代码:
\end{quote}

INSERT INTO student (sno,sname,sex,birthday,dno,spno,classno,entime)

\begin{quote}
values ('200700302','李伟龙','男','880808','001','001','003','20070901')

(注意:要将学号的类型改为char(10),否则长度不够无法执行)

\includegraphics[width=5.75903in,height=4.32292in]{media/image1.png}
\end{quote}

\hypertarget{ux5b9eux9a8c3ux89c6ux56feux4e0eux5b8cux6574ux6027ux7ea6ux675f}{%
\section{实验3《视图与完整性约束》}\label{ux5b9eux9a8c3ux89c6ux56feux4e0eux5b8cux6574ux6027ux7ea6ux675f}}

实验学时: { 4}

每组人数: { 1}

实验类型: { 2} (1:基础性 2:综合性 3:设计性 4:研究性)

实验要求: { 1} (1:必修 2:选修 3:其它)

实验类别: { 3} (1:基础 2:专业基础 3:专业 4:其它)

{}

一、实验目的

学习灵活熟练的进行视图的操作,认识视图的作用;掌握完整性控制的方法。

1.熟悉和掌握对数据表中视图的定义操作和SQL命令的使用;

2.熟悉和掌握对数据表中视图的查询操作和SQL命令的使用;

3.熟悉和掌握对数据表中视图的更新操作和SQL命令的使用,并注意视图更新与基本表更新的区别与联系;

4.熟悉SQL语言对数据库进行完整性控制的方法。

二、实验内容

(一)以S , C , SC表为基础完成以下视图定义及使用

1.定义``SSCH''院学生基本情况视图V\_SSCH;

2.将S,C,SC表中学生的学号,姓名,课程号,课程名,成绩定义为视图V\_S\_C\_G;

3.将各院学生人数,平均年龄定义为视图V\_NUM\_AVG;

4.将各位学生选修课程的门数及平均成绩定义为视图V\_AVG\_S\_G并查询结果;

5.查询平均成绩为90分以上的学生学号、姓名和成绩;

6.通过视图V\_SSCH,新增加一个学生记录 ('S12','YAN XI',19,
'SSCH'),并查询结果;

7.通过视图V\_SSCH,删除学号为``S12''学生信息,并查询结果;

8.将视图V\_SSCH中学号为``S12''的学生改名``中南人''。

(二)使用SQL进行数据完整性控制:包括三类完整性、check短语、constrain子句。

1.
在创建下列关系表时完成如下约束:定义实体完整性;参照完整性(外码、在删除S中的元组时级联删除SC中相应元组、当更新S中的Sno时同时更新SC中的Sno);用户定义完整性:学生年龄\textless30。

2. 修改S中的约束条件,学号在100-1000之间。

学生关系表S :

\begin{longtable}[]{@{}lllll@{}}
\toprule
\endhead
\begin{minipage}[t]{0.17\columnwidth}\raggedright
\begin{quote}
学号
\end{quote}\strut
\end{minipage} & \begin{minipage}[t]{0.17\columnwidth}\raggedright
\begin{quote}
姓名
\end{quote}\strut
\end{minipage} & \begin{minipage}[t]{0.17\columnwidth}\raggedright
\begin{quote}
性别
\end{quote}\strut
\end{minipage} & \begin{minipage}[t]{0.17\columnwidth}\raggedright
\begin{quote}
年龄
\end{quote}\strut
\end{minipage} & \begin{minipage}[t]{0.17\columnwidth}\raggedright
所在系\strut
\end{minipage}\tabularnewline
\begin{minipage}[t]{0.17\columnwidth}\raggedright
\begin{quote}
Sno
\end{quote}\strut
\end{minipage} & \begin{minipage}[t]{0.17\columnwidth}\raggedright
\begin{quote}
Sname
\end{quote}\strut
\end{minipage} & \begin{minipage}[t]{0.17\columnwidth}\raggedright
\begin{quote}
Ssex
\end{quote}\strut
\end{minipage} & \begin{minipage}[t]{0.17\columnwidth}\raggedright
Sage\strut
\end{minipage} & \begin{minipage}[t]{0.17\columnwidth}\raggedright
sdept\strut
\end{minipage}\tabularnewline
\bottomrule
\end{longtable}

~ 创建课程关系表C :

\begin{longtable}[]{@{}llll@{}}
\toprule
\endhead
\begin{minipage}[t]{0.22\columnwidth}\raggedright
\begin{quote}
课程号
\end{quote}\strut
\end{minipage} & \begin{minipage}[t]{0.22\columnwidth}\raggedright
\begin{quote}
课程名
\end{quote}\strut
\end{minipage} & \begin{minipage}[t]{0.22\columnwidth}\raggedright
\begin{quote}
先行课
\end{quote}\strut
\end{minipage} & \begin{minipage}[t]{0.22\columnwidth}\raggedright
\begin{quote}
学分
\end{quote}\strut
\end{minipage}\tabularnewline
\begin{minipage}[t]{0.22\columnwidth}\raggedright
Cno\strut
\end{minipage} & \begin{minipage}[t]{0.22\columnwidth}\raggedright
\begin{quote}
Cname
\end{quote}\strut
\end{minipage} & \begin{minipage}[t]{0.22\columnwidth}\raggedright
\begin{quote}
Cpno
\end{quote}\strut
\end{minipage} & \begin{minipage}[t]{0.22\columnwidth}\raggedright
\begin{quote}
ccredit
\end{quote}\strut
\end{minipage}\tabularnewline
\bottomrule
\end{longtable}

创建学生-课程表SC :

\begin{longtable}[]{@{}lll@{}}
\toprule
\endhead
\begin{minipage}[t]{0.30\columnwidth}\raggedright
\begin{quote}
学号
\end{quote}\strut
\end{minipage} & \begin{minipage}[t]{0.30\columnwidth}\raggedright
\begin{quote}
课程号
\end{quote}\strut
\end{minipage} & \begin{minipage}[t]{0.30\columnwidth}\raggedright
\begin{quote}
成绩
\end{quote}\strut
\end{minipage}\tabularnewline
\begin{minipage}[t]{0.30\columnwidth}\raggedright
\begin{quote}
Sno
\end{quote}\strut
\end{minipage} & \begin{minipage}[t]{0.30\columnwidth}\raggedright
\begin{quote}
Cno
\end{quote}\strut
\end{minipage} & \begin{minipage}[t]{0.30\columnwidth}\raggedright
\begin{quote}
grade
\end{quote}\strut
\end{minipage}\tabularnewline
\bottomrule
\end{longtable}

\begin{quote}
3.用实验验证当操作违反了完整性约束时,系统如何处理?

\textbf{问题:}外键与参照主键是否一定要相同?
\end{quote}

三、实验要求:

1.连接到学生-课程数据库

2.复习有关视图操作的SQL语言命令;复习有关完整性约束操作的SQL语言命令

四、实验步骤

1.定义视图、对视图进行查询、修改;

2.进行完整性约束定义、修改;

3. 进行约束违例验证。

五、实验报告

1.完成本项目实验后,完成实验报告。

2.实验报告格式与要求见附件。

\hypertarget{ux5b9eux9a8c4ux5b58ux50a8ux8fc7ux7a0b}{%
\section{实验4《存储过程》}\label{ux5b9eux9a8c4ux5b58ux50a8ux8fc7ux7a0b}}

实验学时: { 4}

每组人数: { 1}

实验类型: { 2} (1:基础性 2:综合性 3:设计性 4:研究性)

实验要求: { 1} (1:必修 2:选修 3:其它)

实验类别: { 3} (1:基础 2:专业基础 3:专业 4:其它)

{}

一、实验目的

\begin{quote}
理解存储过程的概念、建立和调用方法。进一步熟悉SQL语句对数据库进行完整性控制的方法。
\end{quote}

二、实验内容

1、利用存储过程查找自己的学号、选修课程及成绩。

2、利用存储过程查找姓``李''并且性别为``M''的学生学号、选修课程。

3、利用存储过程计算某同学(学号作为存储过程的参数)所选课程的平均分。

三、实验要求:

了解熟悉存储过程在不同DBMS上实现的差异,熟悉存储过程的写法与调用。

四、实验步骤

1、设计查找自己的学号、选修课程及成绩的存储过程,以自己的学号作为参数,调用存储过程。

2、设计存储过程查找姓``李''并且性别为``M''的学生学号、选修课程并调用。

3、设计存储过程计算某同学(学号作为存储过程的参数)所选课程的平均分并调用。

五、实验报告

1.完成本项目实验后,完成实验报告。

2.实验报告格式与要求见附件。

六、实验示例

\textbf{下面是个销售数据库:}

(1)/*员工人事表employee */

\begin{longtable}[]{@{}lllll@{}}
\toprule
\endhead
emp\_no & char(5) & Not null & primary key & 员工编号\tabularnewline
emp\_name & char(10) & & & 员工姓名\tabularnewline
sex & char(1) & & & 性别\tabularnewline
dept & char(4) & & & 所属部门\tabularnewline
title & char(6) & & & 职称\tabularnewline
date\_hired & datetime & & & 到职日\tabularnewline
birthday & datetime & & & 生日\tabularnewline
salary & int & & & 薪水\tabularnewline
addr & char(50) & null & & 住址\tabularnewline
\bottomrule
\end{longtable}

(2)/*客户表customer */

\begin{longtable}[]{@{}lllll@{}}
\toprule
\endhead
cust\_id & char(5) & Not null & primary key & 客户号\tabularnewline
cust\_name & char(20) & & & 客户名称\tabularnewline
addr & char(40) & & & 客户住址\tabularnewline
tel\_no & char(10) & & & 客户电话\tabularnewline
zip & char(6) & & & 邮政编码\tabularnewline
\bottomrule
\end{longtable}

(3)/*销售主表sales */

\begin{longtable}[]{@{}lllll@{}}
\toprule
\endhead
order\_no & int & Not null & primary key & 订单编号\tabularnewline
cust\_id & char(5) & & & 客户号\tabularnewline
sale\_id & char(5) & & & 业务员编号\tabularnewline
tot\_amt & numeric(9,2) & & & 订单金额\tabularnewline
order\_date & datetime & & & 订货日期\tabularnewline
ship\_date & datetime & & & 出货日期\tabularnewline
invoice\_no & char(10) & & & 发票号码\tabularnewline
\bottomrule
\end{longtable}

(4)/*销货明细表sale\_item */

\begin{longtable}[]{@{}lllll@{}}
\toprule
\endhead
order\_no & int & Not null, & primary key & 订单编号\tabularnewline
prod\_id & char(5) & Not null, & primary key & 产品编号\tabularnewline
qty & int & & & 销售数量\tabularnewline
unit\_price & numeric(7,2) & & & 单价\tabularnewline
order\_date & datetime & null & & 订单日期\tabularnewline
\bottomrule
\end{longtable}

(5)/*产品名称表product */

\begin{longtable}[]{@{}lllll@{}}
\toprule
\endhead
pro\_id & char(5) & Not null & primary key & 产品编号\tabularnewline
prod\_name & char(20) & Not null & & 产品名称\tabularnewline
\bottomrule
\end{longtable}

1、模糊查询

create procedure sp\_empname @E\_name varchar(10) as

select a.emp\_name,a.dept,b.tot\_amt

from employee a inner join sales b

on a.emp\_no=b.sale\_id

where a.emp\_name like @E\_name

go

exec sp\_empname '陈\%'

\begin{enumerate}
\def\labelenumi{\arabic{enumi}.}
\setcounter{enumi}{1}
\item
  利用存储过程计算出'E0014'业务员的销售总金额。
\end{enumerate}

create procedure sp\_saletot @E\_no char(5),@p\_tot int output as

select @p\_tot=sum(tot\_amt)

from sales

where sale\_id=@E\_no

go

declare @tot\_amt int

exec sp\_saletot E0014, @tot\_amt output

select @tot\_amt

\hypertarget{ux5b9eux9a8c5ux89e6ux53d1ux5668ux4e0eux6e38ux6807}{%
\section{实验5《触发器与游标》}\label{ux5b9eux9a8c5ux89e6ux53d1ux5668ux4e0eux6e38ux6807}}

实验学时: { 4}

每组人数: { 1}

实验类型: { 2} (1:基础性 2:综合性 3:设计性 4:研究性)

实验要求: { 1} (1:必修 2:选修 3:其它)

实验类别: { 3} (1:基础 2:专业基础 3:专业 4:其它)

{}

一、实验目的

进一步熟悉SQL语句对数据库进行完整性控制的方法;理解触发器的概念、定义方法和触发条件。理解游标的定义、打开、使用、关闭与释放的方法。

二、实验内容

\begin{enumerate}
\def\labelenumi{\arabic{enumi}.}
\item
  设置一个触发器,该触发器仅允许``dbo''用户可以删除学生表内数据。
\item
  针对学生表写一个DELETE触发器。
\item
  针对学生表写一个UPDATE触发器。
\item
  统计学生的平均成绩,输出低于平均分的成绩(使用游标)。
\end{enumerate}

三、实验要求:

1.熟悉DBMS 工作环境;

2.复习有关SQL语句对数据库进行完整性控制的方法;复习触发器的概念、定义方法和触发条件。复习游标的定义、打开、使用、关闭与释放的方法约束与存储过程的SQL语言命令。

四、实验步骤

1.设置一个触发器,该触发器仅允许``dbo''用户可以删除学生表内数据,否则出错。

2.针对学生表写一个DELETE触发器,删除数据,测试触发。

3.针对学生表写一个UPDATE触发器,更新数据,测试触发。

4.定义游标,统计学生的平均成绩,输出低于平均分的成绩。

五、实验报告

1.完成本项目实验后,完成实验报告。

2.实验报告格式与要求见附件。

六、实验示例

有员工人事表employee

\begin{longtable}[]{@{}lllll@{}}
\toprule
\endhead
emp\_no & char(5) & Not null & primary key & 员工编号\tabularnewline
emp\_name & char(10) & Not null & & 员工姓名\tabularnewline
sex & char(1) & Not null & & 性别\tabularnewline
dept & char(4) & Not null & & 所属部门\tabularnewline
title & char(6) & Not null & & 职称\tabularnewline
date\_hired & datetime & Not null & & 到职日\tabularnewline
birthday & datetime & Null & & 生日\tabularnewline
salary & int & Not null & & 薪水\tabularnewline
addr & char(50) & null & & 住址\tabularnewline
\bottomrule
\end{longtable}

销货明细表sale\_item

\begin{longtable}[]{@{}lllll@{}}
\toprule
\endhead
order\_no & int & Not null, & primary key & 订单编号\tabularnewline
prod\_id & char(5) & Not null, & primary key & 产品编号\tabularnewline
qty & int & Not null & & 销售数量\tabularnewline
unit\_price & numeric(7,2) & Not null & & 单价\tabularnewline
order\_date & datetime & null & & 订单日期\tabularnewline
\bottomrule
\end{longtable}

1、写一个允许用户一次只删除一条记录的触发器。

create trigger tr\_emp

on employee for delete as

declare @row\_cnt int

select @row\_cnt=count(*) from deleted

if @row\_cnt\textgreater1

begin

print '此删除操作可能会删除多条人事表数据!!!'

rollback transaction

end

delete from employee

where sex='女' /*结果:所影响的行数为0*/

2、显示女职员的工号,姓名

\begin{quote}
(1)声明一个游标

语法:DECLARE \textless 游标名\textgreater{} CURSOR FOR
\textless SELECT语句\textgreater;

(2)打开该游标;

语法:open \textless 游标名\textgreater{}

(3)将该游标的内容提取到变量中;

Fetch \textless 游标名\textgreater{} Into \textless 主变量\textgreater{}
\end{quote}

declare cur\_emp\_gender scroll cursor for---定义游标

select emp\_no ,emp\_name

from employee

where sex='F'

order by emp\_no

open cur\_emp\_gender---打开游标

declare @no,@name

fetch next from cur\_emp\_gender into @no,@name---读数据到局部变量

while(@@FETCH\_STATUS=0)

begin

select @no,@name

fetch next from cur\_emp\_gender into @no,@namer

end

close cur\_emp\_gender---关闭

deallocate cur\_emp\_gender---释放

\textbf{附录1:实验报告格式与要求}

\begin{longtable}[]{@{}l@{}}
\toprule
\endhead
\begin{minipage}[t]{0.97\columnwidth}\raggedright
\textbf{封面}

\textbf{《数据库系统SSD7》}

\textbf{实验报告}

专业班级 { }

学 号 { }

姓 名 { }

{}

\textbf{实验成绩:}

\textbf{批阅教师:}

年 月 日\strut
\end{minipage}\tabularnewline
\bottomrule
\end{longtable}

\textbf{正文要求}

\hypertarget{ux5b9eux9a8c1xxxxxux9879ux76eeux540dux79f0}{%
\section{实验1《XXXXX(项目名称)》}\label{ux5b9eux9a8c1xxxxxux9879ux76eeux540dux79f0}}

实验学时: { } 实验地点: { } 实验日期: { }

{}

\textbf{一、实验目的}

指出此次实验应该达到的学习目标。

\textbf{二、实验内容}

指出此次实验应完成的任务。

\textbf{三、实验方法与实验步骤}

包括实验方法、原理、技术、方案等。指出完成该实验的操作步骤。

\textbf{四、实验结果}

记录实验输出数据和结果;对实验数据和结果进行分析描述,给出实验取得的成果和结论。

注:有程序的要求附上程序源代码(不能截图),有图表的要有截图并有相应的文字说明和分析

\textbf{五、实验小结}

给出本次实验的体会,如学会了什么,遇到哪些问题,如何解决这些问题,存在哪些有待改进的地方。

\textbf{附录2 SQL Server 2005的使用}

\textbf{0.启动SQL server 服务:在程序菜单上选Sql server 2005 的配置管理}

\includegraphics[width=6.62431in,height=2.57431in]{media/image2.png}

\includegraphics[width=6.49792in,height=4.56111in]{media/image3.png}

1.在程序菜单中启动SQL sever 2005 express

\includegraphics[width=4.74653in,height=2.81528in]{media/image4.png}

2.点击``连接对象资源管理器''按钮,连接服务器(如设置开机就连接的无需做此步)

\includegraphics[width=3.37778in,height=1.67431in]{media/image5.png}

\includegraphics[width=5.53125in,height=3.73958in]{media/image6.png}

3.新建数据库,修改相关参数

\includegraphics[width=5.76667in,height=3.48472in]{media/image7.png}

\includegraphics[width=5.7625in,height=4.67569in]{media/image8.png}

\includegraphics[width=5.775in,height=1.61181in]{media/image9.png}

4.新建表,选定新建的数据库,指向``表'',点右键,``新建表'',输入各列名及类型,在右属性框修改表名。

\includegraphics[width=5.75903in,height=4.31944in]{media/image10.png}\includegraphics[width=5.76736in,height=3.33403in]{media/image11.png}

5.点``新建查询'',在编辑框中输入sql语句,完成建表、查询语句的编辑;点``执行按钮''执行,观察结果。点``存盘''按钮可以对sql脚本存盘。

\includegraphics[width=5.76736in,height=4.22292in]{media/image12.png}

也可通过配置管理器来启动数据库服务器。

\includegraphics[width=5.75903in,height=1.37639in]{media/image13.png}

\includegraphics[width=4.875in,height=1.61597in]{media/image14.png}

T-SQL建立数据库:

\textbf{CREATE DATABASE csu\_tcg}

\textbf{ON PRIMARY}

\textbf{( NAME = 'csu-tcg',}

\textbf{FILENAME
='e:\textbackslash MSSQL\textbackslash DATA\textbackslash csu-tcg.mdf' ,
-\/-文件夹要先建好}

\textbf{SIZE = 3072KB , -\/-单位可以M,K}

\textbf{FILEGROWTH = 1024KB )}

\textbf{LOG ON}

\textbf{( NAME = N'csu-tcg\_log',}

\textbf{FILENAME = N'c:\textbackslash Program
Files\textbackslash Microsoft SQL
Server\textbackslash MSSQL.1\textbackslash MSSQL\textbackslash DATA\textbackslash csu-tcg\_log.ldf'
,}

\textbf{SIZE = 1MB ,}

\textbf{FILEGROWTH = 10\%)}

\textbf{GO /*go}为事务提交一段SQL脚本(是隐形事务)的结束标志符。SQL ~
Server ~ 应用程序可将多条 ~ Transact-SQL ~ 语句作为一个批处理发给 ~ SQL
~ Server去执行。在此批处理中的语句编译成一个执行计划。程序员在SQL ~
Server 实用工具中执行特定语句,或生成Transact-SQL 语句脚本在 ~ SQL ~
Server ~ 实用工具中运行,用 ~ GO ~ 来标识批处理的结束*/

\end{document}
