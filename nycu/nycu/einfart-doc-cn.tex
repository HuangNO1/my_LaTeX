\documentclass[classical]{einfart}

\linenumbers % Enable line numbers

%%================================
%% Import toolkit
%%================================
\usepackage{ProjLib}
\usepackage{longtable}  % breakable tables
\usepackage{hologo}     % more TeX logo
\usetikzlibrary{calc}

\usepackage{blindtext}

\UseLanguage{Chinese}

%%================================
%% For typesetting code
%%================================
\usepackage{listings}
\definecolor{maintheme}{RGB}{70,130,180}
\definecolor{forestgreen}{RGB}{21,122,81}
\definecolor{lightergray}{gray}{0.99}
\lstset{language=[LaTeX]TeX,
    keywordstyle=\color{maintheme},
    basicstyle=\ttfamily,
    commentstyle=\color{forestgreen}\ttfamily,
    stringstyle=\rmfamily,
    showstringspaces=false,
    breaklines=true,
    frame=lines,
    backgroundcolor=\color{lightergray},
    flexiblecolumns=true,
    escapeinside={(*}{*)},
    % numbers=left,
    numberstyle=\scriptsize, stepnumber=1, numbersep=5pt,
    % firstnumber=last,
}
\providecommand{\meta}[1]{$\langle${\normalfont\itshape#1}$\rangle$}
\lstset{moretexcs=%
    {linenumbers,nolinenumbers,part,parttext,chapter,section,subsection,subsubsection,frontmatter,mainmatter,backmatter,tableofcontents,href,
    color,NameTheorem,CreateTheorem,proofideanameEN,cref,dnf,needgraph,UseLanguage,UseOtherLanguage,AddLanguageSetting,maketitle,address,curraddr,email,keywords,subjclass,thanks,dedicatory,PLdate,ProjLib,qedhere
    }
}
\lstnewenvironment{code}%
{\setstretch{1.07}\LocallyStopLineNumbers%
\setkeys{lst}{columns=fullflexible,keepspaces=true}%
}
{\ResumeLineNumbers}
\lstnewenvironment{code*}%
{\setstretch{1.07}\LocallyStopLineNumbers%
\setkeys{lst}{numbers=left,columns=fullflexible,keepspaces=true}%
}
{\ResumeLineNumbers}

%%================================
%% tip
%%================================
\usepackage[many]{tcolorbox}
\newenvironment{tip}[1][提示]{%
    \LocallyStopLineNumbers%
    \begin{tcolorbox}[breakable,
        enhanced,
        width = \textwidth,
        colback = paper, colbacktitle = paper,
        colframe = gray!50, boxrule=0.2mm,
        coltitle = black,
        fonttitle = \sffamily,
        attach boxed title to top left = {yshift=-\tcboxedtitleheight/2, xshift=.5cm},
        boxed title style = {boxrule=0pt, colframe=paper},
        before skip = 0.3cm,
        after skip = 0.3cm,
        top = 3mm,
        bottom = 3mm,
        title={\scshape\sffamily #1}]%
}{\end{tcolorbox}\ResumeLineNumbers}

%%================================
%% Names
%%================================
\providecommand{\minimalist}{\textsf{minimalist}}
\providecommand{\minimart}{\textsf{minimart}}
\providecommand{\minimbook}{\textsf{minimbook}}
\providecommand{\einfart}{\textsf{einfart}}
\providecommand{\simplivre}{\textsf{simplivre}}

%%================================
%% Titles
%%================================
\let\LevelOneTitle\section
\let\LevelTwoTitle\subsection
\let\LevelThreeTitle\subsubsection

%%================================
%% Main text
%%================================
\begin{document}

\title{\einfart{},以极简主义风格排版你的文章}
\author{许锦文}
\thanks{对应版本. \texttt{\einfart{} 2021/08/11}}
\email{\href{mailto:ProjLib@outlook.com}{ProjLib@outlook.com}}
\date{2021年8月,北京}

\maketitle

\begin{abstract}
    \einfart{} 是 \minimalist{} 文档类系列的成员之一,其名称取自于德文的 einfach (``简约''),并取了 artikel (``文章'') 的前三个字母组合而成。整个 \minimalist{} 系列包含用于排版文章的 \minimart{}、\einfart{} 以及用于排版书的 \minimbook{}、\simplivre{}。我设计这一系列的初衷是为了撰写草稿与笔记,使之看上去简朴而不简陋。

    \einfart{} 支持英语、法语、德语、意大利语、葡萄牙语、巴西葡萄牙语、西班牙语、简体中文、繁体中文、日文、俄文,并且同一篇文档中这些语言可以很好地协调。由于采用了自定义字体,需要用 \hologo{XeLaTeX} 或 \hologo{LuaLaTeX} 引擎进行编译。

    这篇说明文档即是用 \einfart{} 排版的 (使用了参数 \texttt{classical}),你可以把它看作一份简短的说明与演示。
\end{abstract}


\setcounter{tocdepth}{2}
{\setstretch{1.07}\tableofcontents}


\medskip
\LevelOneTitle*{开始之前}
为了使用这篇文档中提到的文档类,你需要:
\begin{itemize}
    \item 安装一个尽可能新版本的 TeX Live 或 MikTeX 套装,并确保 \texttt{minimalist} 和 \texttt{projlib} 被正确安装在你的 \TeX 封装中。
    \item 下载并安装所需的字体,参考“关于默认字体”这一节。
    \item 熟悉 \LaTeX{} 的基本使用方式,且会用 \hologo{pdfLaTeX}、\hologo{XeLaTeX} 或 \hologo{LuaLaTeX} 编译你的文档。
\end{itemize}


\LevelOneTitle{使用示例}

\LevelTwoTitle{如何加载}

只需要在第一行写:

\begin{code}
\documentclass{einfart}
\end{code}

即可使用 \einfart{} 文档类。请注意,要使用 \hologo{XeLaTeX} 或 \hologo{LuaLaTeX} 引擎才能编译。

\LevelTwoTitle{一篇完整的文档示例}

首先来看一段完整的示例。

\begin{code*}
\documentclass{einfart}
\usepackage{ProjLib}

\UseLanguage{French}

\begin{document}

\title{(*\meta{title}*)}
\author{(*\meta{author}*)}
\date{\PLdate{2022-04-01}}

\maketitle

\begin{abstract}
    Ceci est un résumé. \dnf<(*\meta{some hint}*)>
\end{abstract}
\begin{keyword}
    AAA, BBB, CCC, DDD, EEE
\end{keyword}

\section{Un théorème}

\begin{theorem}\label{thm:abc}
    Ceci est un théorème.
\end{theorem}
Référence du théorème: \cref{thm:abc}

\end{document}
\end{code*}


如果你觉得这个例子有些复杂,不要担心。现在我们来一点点地观察这个例子。

\clearpage
\LevelThreeTitle{初始化部分}

\medskip
\begin{code}
\documentclass{einfart}
\usepackage{ProjLib}
\end{code}

初始化部分很简单:第一行加载文档类 \einfart{},第二行加载 \ProjLib{} 工具箱,以便使用一些附加功能。

\LevelThreeTitle{设定语言}

\medskip
\begin{code}
\UseLanguage{French}
\end{code}

这一行表明文档中将使用法语(如果你的文章中只出现英语,那么可以不需要设定语言)。你也可以在文章中间用同样的方式再次切换语言。支持的语言包括简体中文、繁体中文、日文、英语、法语、德语、西班牙语、葡萄牙语、巴西葡萄牙语、俄语。

对于这一命令的详细说明以及更多相关命令,可以参考后面关于多语言支持的小节。

\LevelThreeTitle{标题,作者信息,摘要与关键词}

\medskip
\begin{code}
\title{(*\meta{title}*)}
\author{(*\meta{author}*)}
\date{\PLdate{2022-04-01}}
\maketitle

\begin{abstract}
    (*\meta{abstract}*)
\end{abstract}
\begin{keyword}
    (*\meta{keywords}*)
\end{keyword}
\end{code}

开头部分是标题和作者信息块。这个例子中给出的是最基本的形式,事实上你还可以这样写:

\begin{code}
\author{(*\meta{author 1}*)}
\address{(*\meta{address 1}*)}
\email{(*\meta{email 1}*)}
\author{(*\meta{author 2}*)}
\address{(*\meta{address 2}*)}
\email{(*\meta{email 2}*)}
...
\end{code}

另外,你还可以采用 \AmS{} 文档类的写法:

\begin{code}
\title{(*\meta{title}*)}
\author{(*\meta{author 1}*)}
\address{(*\meta{address 1}*)}
\email{(*\meta{email 1}*)}
\author{(*\meta{author 2}*)}
\address{(*\meta{address 2}*)}
\email{(*\meta{email 2}*)}
\date{\PLdate{2022-04-01}}
\subjclass{*****}
\keywords{(*\meta{keywords}*)}

\begin{abstract}
    (*\meta{abstract}*)
\end{abstract}

\maketitle
\end{code}

\LevelThreeTitle{未完成标记}

\medskip
\begin{code}
\dnf<(*\meta{some hint}*)>
\end{code}
当你有一些地方尚未完成的时候,可以用这条指令标记出来,它在草稿阶段格外有用。

\LevelThreeTitle{定理类环境}

\medskip
\begin{code}
\begin{theorem}\label{thm:abc}
    Ceci est un théorème.
\end{theorem}
Référence du théorème: \cref{thm:abc}
\end{code}

常见的定理类环境可以直接使用。在引用的时候,建议采用智能引用 \lstinline|\cref{|\meta{label}\lstinline|}|——这样就不必每次都写上相应环境的名称了。

\begin{tip}
如果你之后想要切换到标准文档类,只需要把前两行换为:

\begin{code}
\documentclass{article}
\usepackage[a4paper,margin=1in]{geometry}
\usepackage[hidelinks]{hyperref}
\usepackage[palatino,amsfashion]{ProjLib}
\end{code}

或者使用 \AmS{} 文档类:

\begin{code}
\documentclass{amsart}
\usepackage[a4paper,margin=1in]{geometry}
\usepackage[hidelinks]{hyperref}
\usepackage[palatino]{ProjLib}
\end{code}

\end{tip}



\LevelOneTitle{关于默认字体}
本文档类中默认使用 Palatino Linotype 作为英文主字体,思源宋体、思源黑体、思源等宽作为中文主字体、无衬线字体以及等宽字体,并部分使用了 Neo Euler 作为数学字体。这些字体需要用户自行下载安装。其中,思源字体系列可在 \url{https://github.com/adobe-fonts} 下载 (推荐下载 Super-OTC 版本,这样下载的体积较小)。Neo Euler可以在 \url{https://github.com/khaledhosny/euler-otf} 下载。在没有安装相应的字体时,将采用TeX Live中自带的字体来代替,效果可能会有所折扣。

另外,还使用了 Source Code Pro 作为英文无衬线字体、New Computer Modern Mono 作为英文等宽字体,以及 Asana Math、Tex Gyre Pagella Math、Latin Modern Math 数学字体中的部分符号。这些字体在 TeX Live 或 MikTeX 中已经提供,无需自行下载安装。



\LevelOneTitle{选项}

\einfart{} 文档类有下面几个选项:

\begin{itemize}
    \item 语言选项 \texttt{EN} / \texttt{english} / \texttt{English}、\texttt{FR} / \texttt{french} / \texttt{French},等等
        \begin{itemize}
            \item 具体选项名称可参见下一节的 \meta{language name}。第一个指定的语言将作为默认语言。
            \item 语言选项不是必需的,其主要用途是提高编译速度。不添加语言选项时效果是一样的,只是会更慢一些。
        \end{itemize}
    \item \texttt{draft} 或 \texttt{fast}
        \begin{itemize}
            \item 你可以使用选项 \verb|fast| 来启用快速但略微粗糙的样式,主要区别是:
            \begin{itemize}
                \item 使用较为简单的数学字体设置;
                \item 不启用超链接;
                \item 启用 \ProjLib{} 工具箱的快速模式。
            \end{itemize}
        \end{itemize}
    \begin{tip}
        在文章的撰写阶段,建议使用 \verb|fast| 选项以加快编译速度,改善写作时的流畅度。使用 \verb|fast| 模式时会有“DRAFT”字样的水印,以提示目前处于草稿阶段。
    \end{tip}
    \item \texttt{a4paper} 或 \texttt{b5paper}
        \begin{itemize}
            \item 可选的纸张大小。默认的纸张大小为 7in $\times$ 10in。
        \end{itemize}
    \item \texttt{palatino}、\texttt{times}、\texttt{garamond}、\texttt{noto}、\texttt{biolinum} ~$|$~ \texttt{useosf}
        \begin{itemize}
            \item 字体选项。顾名思义,会加载相应名称的字体。
            \item \texttt{useosf} 选项用来启用“旧式”数字。
        \end{itemize}
    \item \texttt{allowbf}
        \begin{itemize}
            \item 允许加粗。启用这一选项时,题目、各级标题、定理类环境名称会被加粗。
        \end{itemize}
\clearpage
    \item \texttt{classical}
        \begin{itemize}
            \item 经典模式。使用这一选项时,将会启用较为古色古香的风格,如同当前的这篇说明文档一样。
        \end{itemize}
    \item \texttt{useindent}
        \begin{itemize}
            \item 采用段首缩进而不是段间间距。
        \end{itemize}
    \item \texttt{runin}
        \begin{itemize}
            \item \lstinline|\subsubsection| 采用 ``runin'' 风格。
        \end{itemize}
    \item \texttt{puretext} 或 \texttt{nothms}
        \begin{itemize}
            \item 纯文本模式,不加载定理类环境。
        \end{itemize}
    \item \texttt{delaythms}
        \begin{itemize}
            \item 将定理类环境设定推迟到导言结尾。如果你希望定理类环境跟随自定义计数器编号,则应考虑这一选项。
        \end{itemize}
    \item \texttt{nothmnum}、\texttt{thmnum} 或 \texttt{thmnum=}\meta{counter}
        \begin{itemize}
            \item 定理类环境均不编号 / 按照 1、2、3 顺序编号 / 在 \meta{counter} 内编号。其中 \meta{counter} 应该是自带的计数器 (如 \texttt{subsection}) 或在导言部分自定义的计数器 (在启用 \texttt{delaythms} 选项的情况下)。在没有使用任何选项的情况下将按照 \texttt{chapter} (书) 或 \texttt{section} (文章) 编号。
        \end{itemize}
    \item \texttt{regionalref}、\texttt{originalref}
        \begin{itemize}
            \item 在智能引用时,定理类环境的名称是否随当前语言而变化。默认为 \texttt{regionalref},即引用时采用当前语言对应的名称;例如,在中文语境中引用定理类环境时,无论原环境处在什么语境中,都将使用名称“定理、定义……”。若启用 \texttt{originalref},则引用时会始终采用定理类环境所处语境下的名称;例如,在英文语境中书写的定理,即使稍后在中文语境下引用时,仍将显示为 Theorem。
            \item 在 \texttt{fast} 模式下,\texttt{originalref} 将不起作用。
        \end{itemize}
\end{itemize}

\LevelOneTitle{具体说明}

\LevelTwoTitle{语言设置}

\einfart{} 提供了多语言支持,包括英语、法语、德语、意大利语、葡萄牙语、巴西葡萄牙语、西班牙语、简体中文、繁体中文、日文、俄文。可以通过下列命令来选定语言:
\begin{itemize}
    \item \lstinline|\UseLanguage{|\meta{language name}\lstinline|}|,用于指定语言,在其后将使用对应的语言设定。
    \begin{itemize}
        \item 既可以用于导言部分,也可以用于正文部分。在不指定语言时,默认选定 “English”。
    \end{itemize}
    \item \lstinline|\UseOtherLanguage{|\meta{language name}\lstinline|}{|\meta{content}\lstinline|}|,用指定的语言的设定排版 \meta{content}。
    \begin{itemize}
        \item 相比 \lstinline|\UseLanguage|,它不会对行距进行修改,因此中西文字混排时能保持行距稳定。
    \end{itemize}
\end{itemize}

\meta{language name} 有下列选择 (不区分大小写,如 \texttt{French} 或 \texttt{french} 均可):
\begin{itemize}
    \item 简体中文:\texttt{CN}、\texttt{Chinese}、\texttt{SChinese} 或 \texttt{SimplifiedChinese}
    \item 繁体中文:\texttt{TC}、\texttt{TChinese} 或 \texttt{TraditionalChinese}
    \item 英文:\texttt{EN} 或 \texttt{English}
    \item 法文:\texttt{FR} 或 \texttt{French}
    \item 德文:\texttt{DE}、\texttt{German} 或 \texttt{ngerman}
    \item 意大利语:\texttt{IT} 或 \texttt{Italian}
    \item 葡萄牙语:\texttt{PT} 或 \texttt{Portuguese}
    \item 巴西葡萄牙语:\texttt{BR} 或 \texttt{Brazilian}
    \item 西班牙语:\texttt{ES} 或 \texttt{Spanish}
    \item 日文:\texttt{JP} 或 \texttt{Japanese}
    \item 俄文:\texttt{RU} 或 \texttt{Russian}
\end{itemize}

另外,还可以通过下面的方式来填加相应语言的设置:
\begin{itemize}
    \item \lstinline|\AddLanguageSetting{|\meta{settings}\lstinline|}|
    \begin{itemize}
        \item 向所有支持的语言增加设置 \meta{settings}。
    \end{itemize}
    \item \lstinline|\AddLanguageSetting(|\meta{language name}\lstinline|){|\meta{settings}\lstinline|}|
    \begin{itemize}
        \item 向指定的语言 \meta{language name} 增加设置 \meta{settings}。
    \end{itemize}
\end{itemize}
例如,\lstinline|\AddLanguageSetting(German){\color{orange}}| 可以让所有德语以橙色显示(当然,还需要再加上 \lstinline|\AddLanguageSetting{\color{black}}| 来修正其他语言的颜色)。

\LevelTwoTitle{定理类环境及其引用}

定义、定理等环境已经被预定义,可以直接使用。

具体来说,预设的定理类环境包括:
\texttt{assumption}、\texttt{axiom}、\texttt{conjecture}、\texttt{convention}、\texttt{corollary}、\texttt{definition}、\texttt{definition-proposition}、\texttt{definition-theorem}、\texttt{example}、\texttt{exercise}、\texttt{fact}、\texttt{hypothesis}、\texttt{lemma}、\texttt{notation}、\texttt{observation}、\texttt{problem}、\texttt{property}、\texttt{proposition}、\texttt{question}、\texttt{remark}、\texttt{theorem},以及相应的带有星号 \lstinline|*| 的无编号版本。

在引用定理类环境时,建议使用智能引用 \lstinline|\cref{|\meta{label}\lstinline|}|。这样就不必每次都写上相应环境的名称了。

\medskip
\begin{tip}[例子]
\begin{code}
\begin{definition}[奇异物品] \label{def: strange} ...
\end{code}
将会生成
\begin{definition}[奇异物品]\label{def: strange}
    这是奇异物品的定义。定理类环境的前后有一行左右的间距。在定义结束的时候会有一个符号来标记。
\end{definition}

\lstinline|\cref{def: strange}| 会显示为:\cref{def: strange}。

\medskip
使用 \lstinline|\UseLanguage{English}| 后,定理会显示为:

\UseLanguage{English}
\begin{theorem}[Useless]\label{thm}
    A theorem in English.
\end{theorem}

默认情况下,引用时,定理类环境的名称总是与当前语言相匹配,例如,上面的定义在现在的英文模式下将显示为英文:\cref{def: strange,thm}。如果在引用时想让定理的名称总是与原定理所在区域的语言匹配,即总是显示原始名称,可以在全局选项中加入 \texttt{originalref}。
\end{tip}

\UseLanguage{Chinese}

\LevelTwoTitle{定义新的定理型环境}

若需要定义新的定理类环境,首先要定义这个环境在所用语言下的名称:
\begin{itemize}
    \item \lstinline|\NameTheorem[|\meta{language name}\lstinline|]{|\meta{name of environment}\lstinline|}{|\meta{name string}\lstinline|}|
\end{itemize}
其中,\meta{language name} 可参阅关于语言设置的小节。当不指定 \meta{language name}时,则会将该名称设置为所有支持语言下的名称。另外,带星号与不带星号的同名环境共用一个名称,因此 \lstinline|\NameTheorem{envname*}{...}| 与 \lstinline|\NameTheorem{envname}{...}| 效果相同。

然后用下面五种方式之一定义这一环境:
\begin{itemize}
    \item \lstinline|\CreateTheorem*{|\meta{name of environment}\lstinline|}|
        \begin{itemize}
            \item 定义不编号的环境 \meta{name of environment}
        \end{itemize}
    \item \lstinline|\CreateTheorem{|\meta{name of environment}\lstinline|}|
        \begin{itemize}
            \item 定义编号环境 \meta{name of environment},按顺序编号
        \end{itemize}
    \item \lstinline|\CreateTheorem{|\meta{name of environment}\lstinline|}[|\meta{numbered like}\lstinline|]|
        \begin{itemize}
            \item 定义编号环境 \meta{name of environment},与 \meta{numbered like} 计数器共用编号
        \end{itemize}
    \item \lstinline|\CreateTheorem{|\meta{name of environment}\lstinline|}<|\meta{numbered within}\lstinline|>|
        \begin{itemize}
            \item 定义编号环境 \meta{name of environment},在 \meta{numbered within} 计数器内编号
        \end{itemize}
    \item \lstinline|\CreateTheorem{|\meta{name of environment}\lstinline|}(|\meta{existed environment}\lstinline|)|\\
    \lstinline|\CreateTheorem*{|\meta{name of environment}\lstinline|}(|\meta{existed environment}\lstinline|)|
        \begin{itemize}
            \item 将 \meta{name of environment} 与 \meta{existed environment} 或 \meta{existed environment}\lstinline|*| 等同。
            \item 这种方式通常在两种情况下比较有用:
                \begin{enumerate}
                    \item 希望定义更简洁的名称。例如,使用 \lstinline|\CreateTheorem{thm}(theorem)|,便可以直接用名称 \texttt{thm} 来撰写定理。
                    \item 希望去除某些环境的编号。例如,使用 \lstinline|\CreateTheorem{remark}(remark*)|,便可以去除 \texttt{remark} 环境的编号。
                \end{enumerate}
        \end{itemize}
\end{itemize}

\begin{tip}
    其内部使用了 \textsf{amsthm},因此传统的 \texttt{theoremstyle} 对其也是适用的,只需在相关定义前标明即可。
\end{tip}

\NameTheorem[CN]{proofidea}{思路}
\CreateTheorem*{proofidea*}
\CreateTheorem{proofidea}<subsection>

\clearpage
下面提供一个例子。这三行代码:

\begin{code}
\NameTheorem[CN]{proofidea}{思路}
\CreateTheorem*{proofidea*}
\CreateTheorem{proofidea}<subsection>
\end{code}

可以分别定义不编号的环境 \lstinline|proofidea*| 和编号的环境 \lstinline|proofidea| (在 subsection 内编号),它们支持在简体中文语境中使用,效果如下所示:

\vspace{-.3\baselineskip}
\begin{proofidea*}
    \lstinline|proofidea*| 环境。
\end{proofidea*}
\vspace{-.5\baselineskip}
\begin{proofidea}
    \lstinline|proofidea| 环境。
\end{proofidea}

\LevelTwoTitle{未完成标记}

你可以通过 \lstinline|\dnf| 来标记尚未完成的部分。例如:
\begin{itemize}
    \item \lstinline|\dnf| 或 \lstinline|\dnf<...>|。效果为:\dnf~或 \dnf<...>。\\其提示文字与当前语言相对应,例如,在法语模式下将会显示为 \UseOtherLanguage{French}{\dnf}。
\end{itemize}

类似的,还有 \lstinline|\needgraph| :
\begin{itemize}
    \item \lstinline|\needgraph| 或 \lstinline|\needgraph<...>|。效果为:\needgraph~或 \needgraph<...>其提示文字与当前语言相对应,例如,在法语模式下将会显示为 \UseOtherLanguage{French}{\needgraph}
\end{itemize}

\LevelTwoTitle{文章标题、摘要与关键词}

\einfart{} 同时具有标准文档类与\AmS{} 文档类的一些特性。

因此,文章的标题部分既可以按照标准文档类 \textsf{article} 的写法来写:

\begin{code}
\title{(*\meta{title}*)}
\author{(*\meta{author}*)\thanks{(*\meta{text}*)}}
\date{(*\meta{date}*)}
\maketitle
\begin{abstract}
    (*\meta{abstract}*)
\end{abstract}
\begin{keyword}
    (*\meta{keywords}*)
\end{keyword}
\end{code}

也可以按照 \AmS{} 文档类的方式来写:

\begin{code}
\title{(*\meta{title}*)}
\author{(*\meta{author}*)}
\thanks{(*\meta{text}*)}
\address{(*\meta{address}*)}
\email{(*\meta{email}*)}
\date{(*\meta{date}*)}
\keywords{(*\meta{keywords}*)}
\subjclass{(*\meta{subjclass}*)}
\begin{abstract}
    (*\meta{abstract}*)
\end{abstract}
\maketitle
\end{code}

作者信息可以包含多组,输入方式为:

\begin{code}
\author{(*\meta{author 1}*)}
\address{(*\meta{address 1}*)}
\email{(*\meta{email 1}*)}
\author{(*\meta{author 2}*)}
\address{(*\meta{address 2}*)}
\email{(*\meta{email 2}*)}
...
\end{code}

其中 \lstinline|\address|、\lstinline|\curraddr|、\lstinline|\email| 的相互顺序是不重要的。

\LevelTwoTitle{其他}

\LevelThreeTitle{关于行号}
行号可以随时开启和关闭。\lstinline|\linenumbers| 用来开启行号,\lstinline|\nolinenumbers| 用来关闭行号。标题、目录、索引等位置为了美观,不进行编号。

\LevelThreeTitle{关于标题中的脚注}
在 \lstinline|\section| 或 \lstinline|\subsection| 中,如果想使用脚注,只能:
\begin{itemize}
    \item 先写 \lstinline|\mbox{\protect\footnotemark}|,
    \item 再在后面用 \lstinline|\footnotetext{...}|。
\end{itemize}
这是标题使用下划线装饰之后带来的一个缺点。

\LevelThreeTitle{关于QED符号}
由于定理类环境中的字体和正文字体是一样的,为了方便看出定理类环境在哪里结束,在其结尾处放置了一个空心的QED符号。然而,如果你的定理是由公式或者列表结尾的,这个符号就无法自动地放在正确的位置。这时,你需要手动在公式或列表最后一个条目的后面加上 \lstinline|\qedhere|,以让QED符号显示到这一行的最后。

\clearpage
\LevelOneTitle{目前存在的问题}

\begin{itemize}[itemsep=.6em]
    \item 对于字体的设置仍然不够完善。
    \item 由于很多核心功能建立在 \ProjLib{} 工具箱的基础上,因此 \minimalist{} (进而 \minimart{}、\einfart{} 与 \minimbook{}、\simplivre{}) 自然继承了其所有问题。详情可以参阅 \ProjLib{} 用户文档的“目前存在的问题”这一小节。
    \item 错误处理功能不完善,在出现一些问题时没有相应的错误提示。
    \item 代码中仍有许多可优化之处。
\end{itemize}


\end{document}
\endinput
%%
%% End of file `einfart/einfart-doc-cn.tex'.
