\documentclass[classical]{einfart}
\usepackage{ProjLib}
\usepackage{hologo}
\usepackage{graphicx} %插入图片的宏包
\usepackage{float} %设置图片浮动位置的宏包
\usepackage{subfigure} %插入多图时用子图显示的宏包
\usepackage{hyperref}
\setCJKmainfont{Noto Serif CJK TC} % 主要字體 Noto Serif
%%================================
%% Titles
%%================================
\let\LevelOneTitle\section
\let\LevelTwoTitle\subsection
\let\LevelThreeTitle\subsubsection

\providecommand{\tightlist}{%
  \setlength{\itemsep}{0pt}\setlength{\parskip}{0pt}}

\begin{document}

\setcounter{tocdepth}{2}
{\setstretch{1.07}\tableofcontents}

\newpage

\part{個人能力}

\section{教育背景}
\textbf{中南大學 \quad 長沙 \quad 中國} \hfill 2018-09 $\sim$ 2021-02

計算機學院 \quad 軟件工程 \quad 肄業生

\textbf{中興大學 \quad 台中 \quad 台灣} \hfill 2021-02 $\sim$ 2023-06

電機系 \quad 在讀生


\section{技術背景}

\begin{itemize}[parsep=0.5ex]
  \item My github:\url{https://github.com/lumusen0305}
  \item 编程语言: $Python >= Verilog >= Java  >= VueJs >= C++ >=  Golang >= C\# == R$.
  \item 平台: ArchLinux.
  \item 開發: Vue,  Qt, Flask, SpringBoot, Markdown, LaTeX, Verilog, MySQL etc.
  \item 主修專業課程: 計算機程序設計基礎(C++)、數據結構、Web應用開發技術、算法分析與設計、計算機網絡原理、軟件開發架構平台、數據庫系統SSD7、軟件體系結構、工程數學、電子學、電路學、電磁學、VLSI、類比電路設計。
  \item 專題: AI Meeting,教授:范志鵬 教授。
  \item 曾加入實驗室: 飛控實驗室,教授:戴訓華 副教授。
\end{itemize}

\section{學習成果}
\begin{itemize}[parsep=0.5ex]
  \item GPA  \quad 4.26/4.3 \hfill 入學 $\sim$ 大三下
  \item 中興電機歷年班排  \quad 1/46 \hfill 入學 $\sim$ 大三下
  \item 中興電機歷年系排  \quad 1/93 \hfill 入學 $\sim$ 大三下
\end{itemize}

\section{擔任職務}

\textbf{中南大學軟件工程 \quad 副年級長} \hfill 2018-09 $\sim$ 2021-02

\textbf{中南大學 \quad 程序部-部員} \hfill 2018-09 $\sim$ 2019-09

\textbf{中南大學 \quad 心助會鐵道分會-部員} \hfill 2018-09 $\sim$ 2019-09
\section{獲獎情况}

\textbf{2019 年度台灣、港澳及華僑學生獎學金 \quad 一等獎} \hfill 2019-12

\textbf{Robocode \quad 校级二等獎} \hfill 2018-11
\begin{figure}[H]
  \centering  %图片全局居中
  \subfigure{
  \label{Fig.sub.1}
  \includegraphics[angle=90,scale=0.2]{Robocode.jpg}
  }
  \caption{Robocode獎狀}
  \label{Fig.main}
\end{figure}



\textbf{QST青軟實訓 \quad A} \hfill  2020-11
\begin{figure}[H]
  \centering  %图片全局居中
  \subfigure{
    \label{Fig.sub.1}
    \includegraphics[angle=90,scale=0.2]{Qt.jpg}
    }
    \caption{Qt實訓證明}
    \label{Fig.main}
  \end{figure}
  
  
  \textbf{中南大學 \quad 肄業證書} \hfill 2021-02
  \begin{figure}[H]
    \centering  %图片全局居中
    \subfigure{
      \label{Fig.sub.1}
  \includegraphics[scale=0.56]{肄業證書.png}
  }
  \caption{中南大學肄業證書}
  \label{Fig.main}
\end{figure}

\textbf{大二下 \quad 中興電機書卷獎} \hfill 2021-12
\begin{figure}[H]
  \centering  %图片全局居中
  \subfigure{
  \label{Fig.sub.1}
  \includegraphics[scale=0.62]{2下.jpg}
  }
  \caption{書卷獎}
  \label{Fig.main}
\end{figure}

\textbf{大三上 \quad 中興電機書卷獎} \hfill 2022-7

\begin{figure}[H]
  \centering  %图片全局居中
  \subfigure{
  \label{Fig.sub.1}
  \includegraphics[scale=0.62]{3上.jpg}
  }
  \caption{書卷獎}
  \label{Fig.main}
\end{figure}

\part{項目經驗}

\section{中興大學時期}

\subsection{智能家居 AI Meeting}

此項目為中興電機的專題,我主要負責前後端整合以及開發、硬件對接、AI 模型運用以及對接,詳情
請參考 Github。

\subsection{OpenMIPS CPU (項目學習,非原創)}

此項目為大三下暑假的學習項目,為了提昇 Verilog 水平,我跟著” 自己動手寫 CPU” 學習 CPU 的
設計流程,項目因為硬體需求止步於教學版。

\subsection{Connect6 (項目修改,非原創)}

這個項目是我在 Youtube 上看到的教學,他有一系列的課程,包含DNN 跟 sobel,我跟著教程一步一步的做出這個項目,並且結合上課內容添加了雙人對打模式。效果如下:
\begin{figure}[H]
    \centering  %图片全局居中
    \subfigure[User1 Win]{
    \label{Fig.sub.1}
    \includegraphics[width=0.45\textwidth]{connect6_1.jpg}}
    \subfigure[User2 Win]{
    \label{Fig.sub.2}
    \includegraphics[width=0.45\textwidth]{connect6_2.jpg}}
    \caption{玩家獲勝演示}
    \label{Fig.main}
\end{figure}


教學網址:\url{https://www.youtube.com/watch?v=KGsFbbbc0sI}

\subsection{新海誠風格變換}

這個功能是我用CycleGAN 加上3部新海誠的電影訓練出來的,雖然操作很簡單,但是我對訓練結果非常滿意就放上來了。
\begin{figure}[H]
    \centering  %图片全局居中
    \subfigure[原圖]{
    \label{Fig.sub.1}
    \includegraphics[width=0.3\textwidth]{IMG_20201204_223844.jpg}}
    \subfigure[風格變化圖]{
    \label{Fig.sub.2}
    \includegraphics[width=0.3\textwidth]{IMG_20201204_223844.png}}
    \caption{風格變換演示}
    \label{Fig.main}
\end{figure}
模型網址:\url{https://github.com/eriklindernoren/PyTorch-GAN}
\subsection{漫畫閱讀器 RHeart}

市面上很多漫畫網站都有廣告以及收費,我運用爬蟲以及 Flask 做出基本的 api 搭配 vue 前端來去除
廣告,有時間會用 kotlin 開發手機客戶端。

\section{中南大學時期}

\subsection{軟件創新創業比賽—智能酒保}
此為學校軟件創新創業校級參賽作品,利用 Pytorch 做表情識別,以及 Fer2013 作為訓練樣本,並用
Flask 作為後端,最後雖然沒得獎但是對於 AI 有稍微的接觸也算是種收穫。

\begin{figure}[H]
    \centering  %图片全局居中
    \subfigure[檢測圖]{
    \label{Fig.sub.1}
    \includegraphics[width=0.35\textwidth]{my_face.jpg}}
    \subfigure[檢測結果]{
    \label{Fig.sub.2}
    \includegraphics[width=0.54\textwidth]{my_face_result.png}}
    \caption{判斷模塊演示}
    \label{Fig.main}
\end{figure}

\subsection{Golang聊天室}

基於golang-websocket實現的線上聊天室,是我目前比較滿意的項目
\subsection{JPetStore重構}

基於SpringBoot製作的寵物商店,因為是第一次使用前後端交互成品較為粗造
\subsection{JPetStore}

使用JSP跟後端邏輯去構造的,並非前後端分離,年代已久沒有push到github
\subsection{Unity2D(項目學習,非原創)}

這是我參考 Brackeys的教學 做出來的類似馬力歐的小遊戲,但是只做了一關。也是我第一次接觸C#。使用遊戲引擎開發遊戲真的比Qt方便很多。

教學網址:\url{https://www.youtube.com/c/Brackeys}
\subsection{JavaFX模擬CPU進程調度}

本項目我是用JAVAFX製作GUI並且用JAVAFX的插件”jfoenix”實現練表功能,jfoenix中有一個GUI叫ListView他的概念跟練表相同,所以我將數值放到listView中用以實現CPU調度並將其可視化。
\begin{figure}[H]
    \centering  %图片全局居中
    \subfigure{
    \label{Fig.sub.1}
    \includegraphics[width=0.7\textwidth]{OS.png}}
    \caption{成果展示}
    \label{Fig.main}
\end{figure}
\subsection{寢室監控系統}

基於Flask、websocket實現的寢室監控系統,使用Heroku架設


\end{document}

