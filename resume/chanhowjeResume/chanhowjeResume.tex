\documentclass[12pt, a4paper]{article}
\usepackage{ctex} % 中文的宏包
\usepackage{indentfirst}
\usepackage{graphicx} % 插入图片的宏包
\usepackage{float} % 设置图片浮动位置的宏包
\usepackage{subfigure} % 插入多图时用子图显示宏包
\usepackage{listings} % 代码块宏包
\usepackage{color} % 代码高亮
\usepackage[colorlinks,linkcolor=blue]{hyperref} % URL 包
\usepackage[pdf]{graphviz}
\usepackage{alphalph}
\renewcommand*{\thesubfigure}{(\arabic{subfigure})}
%%%% 利用tikz来定位照片和学校Logo
\usepackage{tikz}
\usetikzlibrary{calc}
\renewcommand{\baselinestretch}{1.2} %定义行间距1.2
\usepackage{titlesec}
\usepackage{enumitem}
% disable indent globally
\setlength{\parindent}{0pt}
% some general improvements, defines the XeTeX logo
\usepackage{xltxtra}
% use hyperlink for email and url
\usepackage{hyperref}
\hypersetup{hidelinks}
\usepackage{url}
\urlstyle{tt}

% use fontawesome
% \newfontfamily\fontawesome[
%     Path=./
% ]{FontAwesome.otf}
\usepackage{fontawesome5}
\usepackage{xparse}

\setlist{noitemsep} % removes spacing from items but leaves space around the whole list
%\setlist{nosep} % removes all vertical spacing within and around the list
\setlist[itemize]{topsep=0.25em, leftmargin=*}
\setlist[enumerate]{topsep=0.25em, leftmargin=*}

\titleformat{\section}         % Customise the \section command 
  {\large\bfseries\raggedright} % Make the \section headers large (\Large),
                               % small capitals (\scshape) and left aligned (\raggedright)
  {}{0em}                      % Can be used to give a prefix to all sections, like 'Section ...'
  {}                           % Can be used to insert code before the heading
  [{\color{CVBlue}\titlerule}]                 % Inserts a horizontal line after the heading
\titlespacing*{\section}{0cm}{*1.6}{*1.2}

\usepackage[
	a4paper,
	left=1.2cm,
	right=1.2cm,
	top=1.5cm,
	bottom=1.2cm,
	nohead
]{geometry}
\definecolor{dkgreen}{rgb}{0,0.6,0}
\definecolor{gray}{rgb}{0.5,0.5,0.5}
\definecolor{mauve}{rgb}{0.58,0,0.82}

\lstset{ %
    %language=Octave,                % the language of the code
    basicstyle=\scriptsize\Hack,           % the size of the fonts that are used for the code
    numbers=none,                   % where to put the line-numbers
    numberstyle=\tiny\color{gray},  % the style that is used for the line-numbers
    stepnumber=2,                   % the step between two line-numbers. If it's 1, each line 
                                    % will be numbered
    numbersep=3pt,                  % how far the line-numbers are from the code
    backgroundcolor=\color{white},      % choose the background color. You must add \usepackage{color}
    showspaces=false,               % show spaces adding particular underscores
    showstringspaces=false,         % underline spaces within strings
    showtabs=false,                 % show tabs within strings adding particular underscores
    frame=single,                   % adds a frame around the code
    rulecolor=\color{black},        % if not set, the frame-color may be changed on line-breaks within not-black text (e.g. commens (green here))
    tabsize=2,                      % sets default tabsize to 2 spaces
    captionpos=b,                   % sets the caption-position to bottom
    breaklines=true,                % sets automatic line breaking
    breakatwhitespace=false,        % sets if automatic breaks should only happen at whitespace
    title=\lstname,                   % show the filename of files included with \lstinputlisting;
                                    % also try caption instead of title
    keywordstyle=\color{blue},          % keyword style
    commentstyle=\color{dkgreen},       % comment style
    stringstyle=\color{mauve},         % string literal style
    escapeinside={\%*}{*},            % if you want to add LaTeX within your code
    morekeywords={*,...}               % if you want to add more keywords to the set
}
\setCJKmainfont{Noto Serif CJK TC} % 主要字体 Noto Serif
\definecolor{CVBlue}{RGB}{23,110,191}
\newfontfamily\Hack{Hack} % 代码字体

\begin{document}
% \pagenumbering{gobble} % suppress displaying page number

%%%% 利用tikz来定位照片,部分招聘单位可能需要“以貌取人”
\begin{tikzpicture}[remember picture, overlay] 
  \node[anchor = north east] at ($(current page.north east)+(-1cm,-1.2cm)$) {\includegraphics[height=2.5cm]{avatar}};
\end{tikzpicture}%
%%%% 利用tikz来定位学校Logo,这里只在第一页显示,如果需要每页都有,可以考虑在页眉、页脚或者background中加入,不过简历也就一两页,无所谓了
\begin{tikzpicture}[remember picture, overlay] 
  \node[anchor = north west] at ($(current page.north west)+(0.2cm,-0.2cm)$) {\includegraphics[height=2cm]{certeral_south_university_logo}};
\end{tikzpicture}%
%%%% 利用tikz来定位页脚栏,电子版简历使用,黑白纸质打印效果可能并不好。这里只在第一页显示,如果需要每页都有,页脚或者background中加入。
% \begin{tikzpicture}[remember picture, overlay] 
%   \node[anchor = south,fill=CVBlue,draw=none,minimum width=\paperwidth ,minimum height=1.5em ,align=center ,font=\footnotesize ,text=white] at ($(current page.south)$) {\faLinkedinSquare https://www.linkedin.com/in/username \qquad \faGithub https://github.com/HuangNO1 \qquad \faRssSquare http://huangno1.github.io/};
% \end{tikzpicture}%
%tikzpicture环境很敏感,注释周围的空格、空行都会引起水平距离或垂直距离的变化,
%

\centerline{\LARGE\bfseries{陈浩杰}}

\centerline{\normalsize{2022应届生,研究方向:前端技术应用开发}}

\centerline{\normalsize{\faPhone\ 183-7495-9321 \quad \faEnvelope\ \href{qqmail:1411307268@qq.com}{1411307268@qq.com}}}

% \section{\makebox[\faGraduationCap][c]{\color{CVBlue}\faGraduationCap}\  教育背景}

\section{\color{CVBlue}\faGraduationCap\  教育背景}

\textbf{中南大学} \hfill 2018-09 $\sim$ 2022-06

计算机学院 \quad 软件工程 \quad 本科生

\textbf{主修课程}

Web 应用开发技术、软件开发架构平台。

\section{\color{CVBlue}\faCogs\ 技术背景}
% increase linespacing [parsep=0.5ex]
\begin{itemize}[parsep=0.5ex]
  \item 编程语言: C++, Java, Python, VB.net, JavaScript.
  \item 开发: JQuery, React, Spring, Markdown, MySQL etc.
  \item 已修专业课程: 计算机程序设计基础(C++)、信息系统基础 SSD1、数据结构、离散结构、Java 面向对象程序设计 SSD3、计算机图形学、用户界面设计与评价 SSD4、Web应用开发技术、操作系统原理、算法分析与设计、数字电子技术、汇编语言程序设计、编译原理、计算机网络原理、软件开发架构平台、软件工程基础、数据库系统SSD7、科学计算与数学建模、软件体系结构、电子商务应用、云计算及应用、软件测试技术、软件需求工程、大型数据库技术。
  \item GPA:3.3。
  \item 证书: 英语 CET-6,能流畅阅读与理解英语技术文档。
\end{itemize}

\section{\color{CVBlue}\faUsers\ 项目经历}

\textbf{微信小程序 - 食堂智能推荐菜单 (WeChat App、E-Charts)}

项目角色: 队长

主要职责与业绩: 本项目为大学生创新创业项目,项目目的是为了实现食堂菜单的智能推荐,我在其中负责了微信小程序的前端开发工作,项目中使用 E-charts 框架将数据可视化图形界面,以及 CRUD 基础功能实现,最终获得国家级项目评定。

\textbf{僵尸企业识别平台 (JQuery、E-Chart、Spring)}

项目角色: 队长

主要职责与业绩: 本项目为中国大学生服务外包创新创业大赛项目,是基于机器学习对僵尸企业识别与交互式的 Web 应用开发,我在项目中负责构建了前端 Web 开发,前端使用了 JQuery 与 Ajax 请求技术,并基于 FormData 实现文件传输。

\textbf{宠物商店 (JavaScript、Thymeleaf、SpringBoot、MySQL)}

项目角色: 队长

主要职责与业绩: 本项目是计算机学院软件工程软件开架构平台课程项目,我负责前端与后端的全栈开发,前端使用 JavaScript 与 Thymeleaf 模板引擎开发前端页面,后端则使用 SpringBoot 框架,数据库使用 MySQL。


\section{\color{CVBlue}\faAtlassian\ 个人总结}

我在大学期间认真学习计算机相关技术,除了基础课程知识外,我还自主学习了前端 React 与 JQuery 框架技术,并使用 Spring 进行后端开发完成各个项目开发,对于 B/S 与 C/S 风格软件开发有所了解,我会继续加强自己的前端技术开发,也希望能在未来的实习与职场获得更多成长。

\end{document}