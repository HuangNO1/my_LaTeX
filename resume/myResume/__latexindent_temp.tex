\documentclass[10pt, a4paper]{article}
\pagestyle{empty} 
\usepackage{ctex} % 中文的宏包
\usepackage{indentfirst}
\usepackage{graphicx} % 插入图片的宏包
\usepackage{float} % 设置图片浮动位置的宏包
\usepackage{subfigure} % 插入多图时用子图显示宏包
\usepackage{listings} % 代码块宏包
\usepackage{color} % 代码高亮
\usepackage[colorlinks,linkcolor=blue]{hyperref} % URL 包
\usepackage[pdf]{graphviz}
\usepackage{alphalph}
\renewcommand*{\thesubfigure}{(\arabic{subfigure})}
%%%% 利用tikz来定位照片和学校Logo
\usepackage{tikz}
\usetikzlibrary{calc}
\renewcommand{\baselinestretch}{1.2} %定义行间距1.2
\usepackage{titlesec}
\usepackage{enumitem}
% disable indent globally
\setlength{\parindent}{0pt}
% some general improvements, defines the XeTeX logo
\usepackage{xltxtra}
% use hyperlink for email and url
\usepackage{hyperref}
\hypersetup{hidelinks}
\usepackage{url}
\urlstyle{tt}

% use fontawesome
% \newfontfamily\fontawesome[
%     Path=./
% ]{FontAwesome.otf}
\usepackage{fontawesome5}
\usepackage{xparse}

\setlist{noitemsep} % removes spacing from items but leaves space around the whole list
%\setlist{nosep} % removes all vertical spacing within and around the list
\setlist[itemize]{topsep=0.25em, leftmargin=*}
\setlist[enumerate]{topsep=0.25em, leftmargin=*}

\titleformat{\section}         % Customise the \section command 
  {\large\bfseries\raggedright} % Make the \section headers large (\Large),
                               % small capitals (\scshape) and left aligned (\raggedright)
  {}{0em}                      % Can be used to give a prefix to all sections, like 'Section ...'
  {}                           % Can be used to insert code before the heading
  [{\color{CVBlue}\titlerule}]                 % Inserts a horizontal line after the heading
\titlespacing*{\section}{0cm}{*1.6}{*1.2}

\usepackage[
	a4paper,
	left=1.2cm,
	right=1.2cm,
	top=1.5cm,
	bottom=1cm,
	nohead
]{geometry}
\definecolor{dkgreen}{rgb}{0,0.6,0}
\definecolor{gray}{rgb}{0.5,0.5,0.5}
\definecolor{mauve}{rgb}{0.58,0,0.82}

\lstset{ %
    %language=Octave,                % the language of the code
    basicstyle=\scriptsize\Hack,           % the size of the fonts that are used for the code
    numbers=none,                   % where to put the line-numbers
    numberstyle=\tiny\color{gray},  % the style that is used for the line-numbers
    stepnumber=2,                   % the step between two line-numbers. If it's 1, each line 
                                    % will be numbered
    numbersep=3pt,                  % how far the line-numbers are from the code
    backgroundcolor=\color{white},      % choose the background color. You must add \usepackage{color}
    showspaces=false,               % show spaces adding particular underscores
    showstringspaces=false,         % underline spaces within strings
    showtabs=false,                 % show tabs within strings adding particular underscores
    frame=single,                   % adds a frame around the code
    rulecolor=\color{black},        % if not set, the frame-color may be changed on line-breaks within not-black text (e.g. commens (green here))
    tabsize=2,                      % sets default tabsize to 2 spaces
    captionpos=b,                   % sets the caption-position to bottom
    breaklines=true,                % sets automatic line breaking
    breakatwhitespace=false,        % sets if automatic breaks should only happen at whitespace
    title=\lstname,                   % show the filename of files included with \lstinputlisting;
                                    % also try caption instead of title
    keywordstyle=\color{blue},          % keyword style
    commentstyle=\color{dkgreen},       % comment style
    stringstyle=\color{mauve},         % string literal style
    escapeinside={\%*}{*},            % if you want to add LaTeX within your code
    morekeywords={*,...}               % if you want to add more keywords to the set
}
\setCJKmainfont{Noto Serif CJK TC} % 主要字体 Noto Serif
\definecolor{CVBlue}{RGB}{23,110,191}
\newfontfamily\Hack{Hack} % 代码字体

\begin{document}
% \pagenumbering{gobble} % suppress displaying page number

%%%% 利用tikz来定位照片,部分招聘单位可能需要“以貌取人”
% \begin{tikzpicture}[remember picture, overlay] 
%   \node[anchor = north east] at ($(current page.north east)+(-1cm,-1.2cm)$) {\includegraphics[height=2.5cm]{avatar}};
% \end{tikzpicture}%
%%%% 利用tikz来定位学校Logo,这里只在第一页显示,如果需要每页都有,可以考虑在页眉、页脚或者background中加入,不过简历也就一两页,无所谓了
\begin{tikzpicture}[remember picture, overlay] 
  \node[anchor = north west] at ($(current page.north west)+(0.2cm,-0.2cm)$) {\includegraphics[height=2cm]{certeral_south_university_logo}};
\end{tikzpicture}%
%%%% 利用tikz来定位页脚栏,电子版简历使用,黑白纸质打印效果可能并不好。这里只在第一页显示,如果需要每页都有,页脚或者background中加入。
% \begin{tikzpicture}[remember picture, overlay] 
%   \node[anchor = south,fill=CVBlue,draw=none,minimum width=\paperwidth ,minimum height=1.5em ,align=center ,font=\footnotesize ,text=white] at ($(current page.south)$) {\faLinkedinSquare https://www.linkedin.com/in/username \qquad \faGithub https://github.com/HuangNO1 \qquad \faRssSquare http://huangno1.github.io/};
% \end{tikzpicture}%
%tikzpicture环境很敏感,注释周围的空格、空行都会引起水平距离或垂直距离的变化,
%

\centerline{\LARGE\bfseries{黄柏曛}}

\centerline{\normalsize{2022应届生,方向:应用开发}}

\centerline{\normalsize{\faPhone\ 186-7312-1200 \quad \faEnvelope\ \href{qqmail:2116280484@qq.com}{2116280484@qq.com}}}

% \section{\makebox[\faGraduationCap][c]{\color{CVBlue}\faGraduationCap}\  教育背景}

\section{\color{CVBlue}\faGraduationCap\  教育背景}

\textbf{中南大学 \quad 长沙 \quad 中国} \hfill 2018-09 $\sim$ 2022-06

计算机学院 \quad 软件工程 \quad 本科生

\section{\color{CVBlue}\faUsers\ 项目经历}

% \textbf{仿 QQ 通讯软件 (Qt C++、Golang、Gorilla WebSocket)} \hfill 2019-07 $\sim$ 2019-08

% 项目角色: \quad 同學合作開發

% 項目Github:\href{https://github.com/junyussh/MumiChat-Client}{Qt 前端}、\href{https://github.com/junyussh/MumiChat-server}{Golang 後端}

% 主要职责与业绩: \quad 此为項目實現桌面端通訊軟件,我使用前端 Qt5 C++ 實現聊天應用窗口的登入、好友列表獲取、聊天信息收發,而后端基于 Golang 的 Gorilla 套件开发實現用戶的 Channel 文字傳輸、好友列表傳輸等。

% 成長:實際將面向對象、數據結構運用在實際應用,像是用戶的名字、郵件、好友列表,進行抽象繼承多態,與 WebSock 通訊協議進行文字傳輸。

% \textbf{基于 JSP 技术实现宠物商店网站 (JSP、Tomcat、MySQL、JQuery)} \hfill 2019-09 $\sim$ 2019-12

% 项目角色: \quad 队长

% 主要职责与业绩: \quad 此为学校课程小组项目,我负责队伍工作调度、前端开发、数据库设计、整体项目监控,完成作品,获老师佳评。

% \textbf{动画网站 (VueJS、Django)} \hfill 2019-10 $\sim$ 2019-12

% 项目角色: \quad 技术开发人员

% 主要职责与业绩: \quad 此为我与一位室友合作开发的项目,他负责后端的数据爬取与接口设计,我负责前端开发与对接,最终完成项目。

% \textbf{基于微空运输的移植器官运输路径规划系统 (VueJS、Flask)} \hfill 2020-01 $\sim$ 2020-10

% 项目角色: \quad 技术开发人员

% 主要职责与业绩: \quad 我负责前端页面开发、后端开发、前后端对接,这是交通院的交通科技大赛与大学生创新创业项目,最终获得交科赛获校级一等奖,原被推荐进国赛;大创获得省级项目评定。

% \textbf{基于前后端完全分离实现宠物商店 (VueJS、SpringBoot、MySQL)} \hfill 2020-03 $\sim$ 2020-06

% 项目角色: \quad 队长

% 主要职责与业绩: \quad 此为学校课程小组项目,我负责队伍工作调度、前端页面开发、前后端接口对接、数据库设计、整体项目监控,实现完整功能並完成作品,获老师佳评。

\textbf{个人空间系统 (VueJS、SpringBoot、Flask、MySQL)} \hfill 2020-07 $\sim$ 2020-08

% 项目角色: \quad 技术开发人员
\begin{itemize}[parsep=0.5ex]
\item \textbf{项目Github}: \href{https://github.com/2892211452/SXCsuOntOf}{https://github.com/2892211452/SXCsuOntOf}

\item \textbf{主要职责与业绩}: \quad 博客系统对应学校学生使用,使用 Vue 实现关于博客平台的Markdown文章编辑、发表、收藏、用户资讯修改等,并使用 Python 爬取新闻资讯整合成Flask功能模块接口,整体后端逻辑业务使用 SpringBoot,MySQL存储用户信息、文章信息、文章Tag等。

\item \textbf{成长}:前端学习到 Vue-cli 的开发工具使用、 vue-route、async/awit处理异步数据请求,后端方面Flask的微服务模块区分学校资讯爬虫与免费游戏资讯爬虫、 Restful API 风格设计标准。最终获得项目个人评优。
\end{itemize}

\textbf{简易版今日头条 (Android Java)} \hfill 2021-05 $\sim$ 2021-06

% 项目角色: \quad 个人开发

\begin{itemize}[parsep=0.5ex]
\item \textbf{项目Github} \href{https://github.com/HuangNO1/TouTiao_Simple_Android_App}{https://github.com/HuangNO1/TouTiao\_Simple\_Android\_App}

\item \textbf{主要职责与业绩}: \quad 安卓应用开发课程项目与字节跳动客户端合作,实现完整业务要求,新闻列表展示、多频道切换、上拉刷新、下拉加载更多、图片缓存与懒加载、遵循 Android Code Style等。

\item \textbf{成长}:通过字节跳动,学习到Android的开发前景、使用Java编写Android需要注意到的页面抽象、在页面数据渲染上的主页面线程与数据请求子线程的区别,针对重复需要渲染的文字卡片等进行抽象组件化。
\end{itemize}

\textbf{智能网关应用开发 (Nanopi、C、Modbus、FTP、Vue、Flask)} \hfill 2021-11 $\sim$ 2022-02

% 项目角色: \quad 项目开发人员
\begin{itemize}[parsep=0.5ex]
\item \textbf{主要职责与业绩}: \quad 使用 Modbus 协议读取多个物联网装置的信号变化至共享內存,针对信号变化异常用 FTP 传输装置的异常日志,最终使用 VueJS、Flask 前后端分离编写网关的网页控制。

\item \textbf{成长}:学习 Modbus 协议透過串口与网口进行物联网装置的命令读取读写。網關的网页控制使用 Flask 框架,並使用 BluePrint 进行模块区分系统资讯、文件管理、应用配置文件、SQLite db数据库管理、进程管理、网卡设置等功能模块,并采用 Rest API对一个接口进行功能区分。Vue 开发根据 Flask 后端进行页面路由组件的页面功能设计区分。
\end{itemize}

\section{\color{CVBlue}\faCogs\ 技术背景}
% increase linespacing [parsep=0.5ex]
\begin{itemize}[parsep=0.5ex]
  \item \textbf{编程语言}: $C++, C, JavaScript, Java, Python$.
  % \item 平台: Windows, ArchLinux.
  \item \textbf{开发}: 熟练使用 VueJS 前端框架、Flask 微服务框架、MySQL 数据库。
  \item \textbf{主修专业课程}: 计算机程序设计基础(C++)、数据结构、Web应用开发技术、算法分析与设计、计算机网络原理、软件开发架构平台、数据库系统SSD7、软件体系结构、机器学习与数据挖掘。
  % \item 已加入实验室: 生物基因与数据挖掘相关研究实验室,教授:邓磊。
\end{itemize}

\section{\color{CVBlue}\faBriefcase\ 实习经历}

\textbf{上海钧盟实业有限公司 \quad 研发工程师} \hfill 2021-09 $\sim$ 至今

\textbf{主要职责与业绩}: \quad 负责大全赛雪龙、ORing、Moxa等客户的智能网关应用开发,使用工业物联网的相关协议结合网页开发技术解决装置的状态的读取、网关系统可视化操作等,将公司旧的Web1.0(基于PHP语言编写、GET POST混用)网关控制页面功能重构设计升级Web2.0(基于Vue2.0与Flask,重整后端接口为 Restful API风格),提高用户的使用体验与功能。

\section{\color{CVBlue}\faHeart\ 获奖情况}

\textbf{全国大学生电子商务“创新、创意及创业”挑战赛 \quad 校级二等奖} \hfill 2021-05

\textbf{交通科技大赛 \quad 校级一等奖} \hfill 2020-01

% 获奖情况: \quad 这次的项目我们小组都尽到努力,尤其是在疫情期间大家互相无法见面讨论,开学后交通院的学长姐也要准备考研和实习,即使这样我们小组没懈怠,最终获得校级一等奖,这次的比赛对我来说的成长意义很大,我很高兴有参加到这次的竞赛。

% \textbf{2019 年度台湾、港澳及华侨学生奖学金 \quad 三等奖} \hfill 2019-12

% 获奖情况: \quad 因为我是台湾的学生,大学以前也都是在台湾学习与成长,但依然没放弃努力,积极学习,并且参加各式各样的比赛,我也经常与陆生合作,我很荣幸我能拿到奖学金来支持我的个人学术方面研究与学习,我帮助很多对于软件工程专业内编程不佳的同学学习,我会继续努力深造自己。

% \section{\color{CVBlue}\faAtlassian\ 个人小结}

% 本人在中南大学从 2018 入学至 2020 大三第二学期期间参加了两次大学生创新创业项目,我也参加了中南大学交通院的 2020 交通科技大赛、服务外包大赛与软件创新大赛,都加强了我的成长与经验。我的个人参与的开发项目可以查看我的 Github (ID: HuangNO1),我的 Github 中不乏我平常练习的内容,也入门了 AngularJS、ReactJS 前端技术开发。之前也为了比赛学了快应用,目前我正在学习微信小程序开发。我在 Github Page 利用 Hugo 打造自己的静态页面 Blog。在 Blog 中我不仅分享了我的开发经验帮助别人进行学习,也让自己能够温故知新,我自己平常是使用 ArchLinux 的 Linux 系统进行开发工作,所以 Blog 中的文章有些相关的内容,其中光是 Blog 的编写就让我熟悉 Git 版本控制和 Markdown 语言。目前我还有关于机器学习与深度学习的文章正在编写,以自己的角度与语言重新解释 ML 与 DL 这些技术的学习,帮助小白入门。最后,我已经在 2020 大三第一学期加入了教授的实验室,并在实验室中学习,教授是专门研究生物基因序列与数据挖掘,实验室每天中午有技术会议与论文拜读,我很高兴有这样的机会让我进入实验室和师兄师姐们学习,也很庆幸我自己在中南大学学到了很多。 

% \section{\color{CVBlue}\faChalkboard\ 自我评价}

% \quad\quad 我平时喜好编程,对自己的项目也具有责任感,与同学合作开发也有良好的沟通协作,我不仅通过许多项目经验磨练自己的能力,也虚心接受别人的指导,我会继续加强自己的能力,并透过职场获得更多的成长与反思。

\section{\color{CVBlue}\faInfo\ 其它}
% increase linespacing [parsep=0.5ex]
\begin{itemize}[parsep=0.5ex]
  \item Blog: \href{http://huangno1.github.io/}{http://huangno1.github.io/}
  \item GitHub: \href{https://github.com/HuangNO1}{https://github.com/HuangNO1}
\end{itemize}

\end{document}