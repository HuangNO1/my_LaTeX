\documentclass[12pt, a4paper]{article}
\usepackage{ctex} % 中文的宏包
\usepackage{indentfirst}
\usepackage{graphicx} % 插入圖片的宏包
\usepackage{float} % 設置圖片浮動位置的宏包
\usepackage{subfigure} % 插入多圖時用子圖顯示宏包
\usepackage{listings} % 代碼塊宏包
\usepackage{color} % 代碼高亮
\usepackage[colorlinks,linkcolor=blue]{hyperref} % URL 包
\usepackage[pdf]{graphviz}
\usepackage{alphalph}
\renewcommand*{\thesubfigure}{(\arabic{subfigure})}
%%%% 利用tikz来定位照片和学校Logo
\usepackage{tikz}
\usetikzlibrary{calc}
\renewcommand{\baselinestretch}{1.2} %定义行间距1.2
\usepackage{titlesec}
\usepackage{enumitem}
% disable indent globally
\setlength{\parindent}{0pt}
% some general improvements, defines the XeTeX logo
\usepackage{xltxtra}
% use hyperlink for email and url
\usepackage{hyperref}
\hypersetup{hidelinks}
\usepackage{url}
\urlstyle{tt}

% use fontawesome
% \newfontfamily\fontawesome[
%     Path=./
% ]{FontAwesome.otf}
\usepackage{fontawesome5}
\usepackage{xparse}

\setlist{noitemsep} % removes spacing from items but leaves space around the whole list
%\setlist{nosep} % removes all vertical spacing within and around the list
\setlist[itemize]{topsep=0.25em, leftmargin=*}
\setlist[enumerate]{topsep=0.25em, leftmargin=*}

\titleformat{\section}         % Customise the \section command 
  {\large\bfseries\raggedright} % Make the \section headers large (\Large),
                               % small capitals (\scshape) and left aligned (\raggedright)
  {}{0em}                      % Can be used to give a prefix to all sections, like 'Section ...'
  {}                           % Can be used to insert code before the heading
  [{\color{CVBlue}\titlerule}]                 % Inserts a horizontal line after the heading
\titlespacing*{\section}{0cm}{*1.6}{*1.2}

\usepackage[
	a4paper,
	left=1.2cm,
	right=1.2cm,
	top=1.5cm,
	bottom=1cm,
	nohead
]{geometry}
\definecolor{dkgreen}{rgb}{0,0.6,0}
\definecolor{gray}{rgb}{0.5,0.5,0.5}
\definecolor{mauve}{rgb}{0.58,0,0.82}

\lstset{ %
    %language=Octave,                % the language of the code
    basicstyle=\scriptsize\Hack,           % the size of the fonts that are used for the code
    numbers=none,                   % where to put the line-numbers
    numberstyle=\tiny\color{gray},  % the style that is used for the line-numbers
    stepnumber=2,                   % the step between two line-numbers. If it's 1, each line 
                                    % will be numbered
    numbersep=3pt,                  % how far the line-numbers are from the code
    backgroundcolor=\color{white},      % choose the background color. You must add \usepackage{color}
    showspaces=false,               % show spaces adding particular underscores
    showstringspaces=false,         % underline spaces within strings
    showtabs=false,                 % show tabs within strings adding particular underscores
    frame=single,                   % adds a frame around the code
    rulecolor=\color{black},        % if not set, the frame-color may be changed on line-breaks within not-black text (e.g. commens (green here))
    tabsize=2,                      % sets default tabsize to 2 spaces
    captionpos=b,                   % sets the caption-position to bottom
    breaklines=true,                % sets automatic line breaking
    breakatwhitespace=false,        % sets if automatic breaks should only happen at whitespace
    title=\lstname,                   % show the filename of files included with \lstinputlisting;
                                    % also try caption instead of title
    keywordstyle=\color{blue},          % keyword style
    commentstyle=\color{dkgreen},       % comment style
    stringstyle=\color{mauve},         % string literal style
    escapeinside={\%*}{*},            % if you want to add LaTeX within your code
    morekeywords={*,...}               % if you want to add more keywords to the set
}
\setCJKmainfont{Noto Serif CJK TC} % 主要字體 Noto Serif
\definecolor{CVBlue}{RGB}{23,110,191}
\newfontfamily\Hack{Hack} % 代碼字體

\begin{document}
% \pagenumbering{gobble} % suppress displaying page number

%%%% 利用tikz来定位照片,部分招聘单位可能需要“以貌取人”
\begin{tikzpicture}[remember picture, overlay] 
  \node[anchor = north east] at ($(current page.north east)+(-1cm,-1.2cm)$) {\includegraphics[height=2.5cm]{avatar}};
\end{tikzpicture}%
%%%% 利用tikz来定位学校Logo,这里只在第一页显示,如果需要每页都有,可以考虑在页眉、页脚或者background中加入,不过简历也就一两页,无所谓了
\begin{tikzpicture}[remember picture, overlay] 
  \node[anchor = north west] at ($(current page.north west)+(0.2cm,-0.2cm)$) {\includegraphics[height=2cm]{certeral_south_university_logo}};
\end{tikzpicture}%
%%%% 利用tikz来定位页脚栏,电子版简历使用,黑白纸质打印效果可能并不好。这里只在第一页显示,如果需要每页都有,页脚或者background中加入。
% \begin{tikzpicture}[remember picture, overlay] 
%   \node[anchor = south,fill=CVBlue,draw=none,minimum width=\paperwidth ,minimum height=1.5em ,align=center ,font=\footnotesize ,text=white] at ($(current page.south)$) {\faLinkedinSquare https://www.linkedin.com/in/username \qquad \faGithub https://github.com/HuangNO1 \qquad \faRssSquare http://huangno1.github.io/};
% \end{tikzpicture}%
%tikzpicture环境很敏感,注释周围的空格、空行都会引起水平距离或垂直距离的变化,
%

\centerline{\LARGE\bfseries{黃柏曛}}

\centerline{\normalsize{2022應屆生,研究方向:Web 應用開發、桌面應用開發、操作系統}}

\centerline{\normalsize{\faPhone\ 186-7312-1200 \quad \faEnvelope\ \href{qqmail:2116280484@qq.com}{2116280484@qq.com}}}

% \section{\makebox[\faGraduationCap][c]{\color{CVBlue}\faGraduationCap}\  教育背景}

\section{\color{CVBlue}\faGraduationCap\  教育背景}

\textbf{中南大學} \hfill 2018-09 $\sim$ 2022-06

計算機學院 \quad 軟件工程 \quad 本科生

\section{\color{CVBlue}\faCogs\ IT 技能}
% increase linespacing [parsep=0.5ex]
\begin{itemize}[parsep=0.5ex]
  \item 編程語言: C/C++, Java, Python, Golang, JavaScript.
  \item 平台: ArchLinux.
  \item 開發: Front end design, Vue, React, Angular, Qt, Flask, SpringBoot, Markdown, LaTeX, MySQL etc.
  \item 已修專業課程: 计算机程序设计基础(C++)、信息系统基础 SSD1、数据结构、离散结构、Java 面向对象程序设计SSD3、计算机图形学、用户界面设计与评价SSD4、Web应用开发技术、操作系统原理、算法分析与设计、数字电子技术、汇编语言程序设计、编译原理、计算机网络原理、软件开发架构平台、软件工程基础、数据库系统SSD7、科学计算与数学建模、软件体系结构、电子商务应用、云计算及应用、软件测试技术、软件需求工程。
  \item 已加入實驗室: 生物醫學基因工程與數據挖掘實驗室,教授:鄧磊。
\end{itemize}

\section{\color{CVBlue}\faUsers\ 項目經歷}

\textbf{基於深度學習的智能健康助手 (Kotlin、Tenserflow、Flask)} \hfill 2019-03 $\sim$ 2020-03

项目角色: \quad 隊長

主要职责与业绩: \quad 此為大學生創新創業項目,我負責分配隊員的工作與項目進度調度、前端App開發、深度學習模型訓練,最後獲得校級評定,與``我最喜歡的展示作品''獎項。

\textbf{仿 QQ 通訊軟件 (Qt C++、Golang、Gorilla WebSocket)} \hfill 2019-07 $\sim$ 2019-08

项目角色: \quad 個人開發

主要职责与业绩: \quad 此為暑假於中山實訓項目,我負責前端 Qt 桌面應用開發、後端基於 Golang 的 webSocket 通信協議開發,最終完成作品。

\textbf{基於 JSP 技術實現寵物商店網站 (JSP、Tomcat、MySQL、JQuery)} \hfill 2019-09 $\sim$ 2019-12

项目角色: \quad 隊長

主要职责与业绩: \quad 此為學校課程小組項目,我負責隊伍工作調度、前端開發、數據庫設計、整體項目監控,完成作品,獲老師佳評。

\textbf{動畫網站 (VueJS、Django)} \hfill 2019-10 $\sim$ 2019-12

项目角色: \quad 技術開發人員

主要职责与业绩: \quad 此為我與一位室友合作開發的項目,他負責後端的數據爬取與接口設計,我負責前端開發與對接,最終完成項目。

\textbf{基於微空運輸的移植器官運輸路徑規劃系統 (VueJS、Flask)} \hfill 2020-01 $\sim$ 2020-10

项目角色: \quad 技術開發人員

主要职责与业绩: \quad 我負責前端頁面開發、後端開發、前後端對接,這是交通院的交通科技大賽與大學生創新創業項目,最終獲得交科賽獲校級一等獎,原被推薦進國賽,後面遺憾沒進國賽;大創獲得省級項目評定。

\textbf{基於前後端完全分離實現寵物商店 (VueJS、SpringBoot、MySQL)} \hfill 2020-03 $\sim$ 2020-06

项目角色: \quad 隊長

主要职责与业绩: \quad 此為學校課程小組項目,我負責隊伍工作調度、前端頁面開發、前後端接口對接、數據庫設計、整體項目監控,完成作品,獲老師佳評。

\textbf{個人空間系統 (VueJS、SpringBoot、Flask、MySQL)} \hfill 2020-07 $\sim$ 2020-08

项目角色: \quad 技術開發人員

主要职责与业绩: \quad 此為暑假實訓項目,我負責前端頁面開發、前後端接口對接、接口初步討論設計,最終獲得暑假實訓項目個人評優。

\textbf{``微圾分''工作室構建環保服務電商平台 (WeChat App)} \hfill 2021-02 $\sim$ 現在

项目角色: \quad 技術開發人員

主要职责与业绩: \quad 此為商學院的全國大學生電子商務“創新、創意及創業”挑戰賽(三創賽)項目、大學生創新創業項目、互聯網+項目,我負責前端微信小程序開發、後端開發、前後端對接,目前正在開發階段。

\section{\color{CVBlue}\faHeart\ 獲獎情况}

\textbf{交通科技大賽 \quad 校級一等獎} \hfill 2020-01

獲獎情況: \quad 這次的項目原本是被推進國賽的,但是很遺憾項目文檔不夠吸引評委,止步於校級一等獎,但是這次的比賽對我來說的成長意義很大,在參加這次比賽之前,我之前參加的學校比賽都沒到這麼高的獎項經驗。

\textbf{2019 年度台灣、港澳及華僑學生獎學金 \quad 三等獎} \hfill 2019-12

獲獎情況: \quad 因為我是台灣的學生,大學以前也都是在台灣學習與成長,但依然沒放棄努力,積極學習,並且參加各式各樣的比賽,我也經常與陸生合作,我很榮幸我能拿到獎學金來支持我的個人學術方面研究與學習,我幫助很多對於軟件工程專業內編程不佳的同學學習,我會繼續努力深造自己。

\section{\color{CVBlue}\faAtlassian\ 個人小結}

本人在中南大學從 2018 入學至 2020 大三第二學期期間參加了兩次大學生創新創業項目,我也參加了中南大學交通院的 2020 交通科技大賽、服務外包大賽與軟件創新大賽,都對我的成長意義重大。我的個人參與的開發項目可以查看我的 Github (ID: HuangNO1),我的 Github 中不乏我平常練習的內容,也入門了 AngularJS、ReactJS 前端技術開發。之前也為了比賽學了快應用,目前我正在學習微信小程序開發。我在 Github Page 利用 Hugo 打造自己的靜態頁面 Blog。在 Blog 中我不僅分享了我的開發經驗幫助別人進行學習,也讓自己能夠溫故知新,我自己平常是使用 ArchLinux 的 Linux 系統進行開發工作,所以 Blog 中的文章有些相關的內容,其中光是 Blog 的編寫就讓我熟悉 Git 版本控制和 Markdown 語言。目前我還有關於機器學習與深度學習的文章正在編寫,以自己的角度與語言重新解釋 ML 與 DL 這些技術的學習,幫助小白入門。最後,我已經在 2020 大三第一學期加入了教授的實驗室,並在實驗室中學習,教授是專門研究生物基因序列與數據挖掘,實驗室每天中午有技術會議與論文拜讀,我很高興有這樣的機會讓我進入實驗室和師兄師姐們學習,也很慶幸我自己在中南大學學到了很多。 

\section{\color{CVBlue}\faInfo\ 其它}
% increase linespacing [parsep=0.5ex]
\begin{itemize}[parsep=0.5ex]
  \item Blog: http://huangno1.github.io/
  \item GitHub: https://github.com/HuangNO1
  \item QQ:2116280484
  \item Telegram:@HuangNO1
  \item E-mail:2116280484@qq.com
\end{itemize}

\end{document}