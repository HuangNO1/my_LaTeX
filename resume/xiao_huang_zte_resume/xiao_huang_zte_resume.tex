\documentclass[10pt, a4paper]{article}
\pagestyle{empty} 
\usepackage{ctex} % 中文的宏包
\usepackage{indentfirst}
\usepackage{graphicx} % 插入图片的宏包
\usepackage{float} % 设置图片浮动位置的宏包
\usepackage{subfigure} % 插入多图时用子图显示宏包
\usepackage{listings} % 代码块宏包
\usepackage{color} % 代码高亮
\usepackage[colorlinks,linkcolor=blue]{hyperref} % URL 包
\usepackage[pdf]{graphviz}
\usepackage{alphalph}
\renewcommand*{\thesubfigure}{(\arabic{subfigure})}
%%%% 利用tikz来定位照片和学校Logo
\usepackage{tikz}
\usetikzlibrary{calc}
\renewcommand{\baselinestretch}{1.2} %定义行间距1.2
\usepackage{titlesec}
\usepackage{enumitem}
% disable indent globally
\setlength{\parindent}{0pt}
% some general improvements, defines the XeTeX logo
\usepackage{xltxtra}
% use hyperlink for email and url
\usepackage{hyperref}
\hypersetup{hidelinks}
\usepackage{url}
\urlstyle{tt}

% use fontawesome
% \newfontfamily\fontawesome[
%     Path=./
% ]{FontAwesome.otf}
\usepackage{fontawesome5}
\usepackage{xparse}

\setlist{noitemsep} % removes spacing from items but leaves space around the whole list
%\setlist{nosep} % removes all vertical spacing within and around the list
\setlist[itemize]{topsep=0.25em, leftmargin=*}
\setlist[enumerate]{topsep=0.25em, leftmargin=*}

\titleformat{\section}         % Customise the \section command 
  {\large\bfseries\raggedright} % Make the \section headers large (\Large),
                               % small capitals (\scshape) and left aligned (\raggedright)
  {}{0em}                      % Can be used to give a prefix to all sections, like 'Section ...'
  {}                           % Can be used to insert code before the heading
  [{\color{CVBlue}\titlerule}]                 % Inserts a horizontal line after the heading
\titlespacing*{\section}{0cm}{*1.6}{*1.2}

\usepackage[
	a4paper,
	left=1.2cm,
	right=1.2cm,
	top=1.5cm,
	bottom=1cm,
	nohead
]{geometry}
\definecolor{dkgreen}{rgb}{0,0.6,0}
\definecolor{gray}{rgb}{0.5,0.5,0.5}
\definecolor{mauve}{rgb}{0.58,0,0.82}

\lstset{ %
    %language=Octave,                % the language of the code
    basicstyle=\scriptsize\Hack,           % the size of the fonts that are used for the code
    numbers=none,                   % where to put the line-numbers
    numberstyle=\tiny\color{gray},  % the style that is used for the line-numbers
    stepnumber=2,                   % the step between two line-numbers. If it's 1, each line 
                                    % will be numbered
    numbersep=3pt,                  % how far the line-numbers are from the code
    backgroundcolor=\color{white},      % choose the background color. You must add \usepackage{color}
    showspaces=false,               % show spaces adding particular underscores
    showstringspaces=false,         % underline spaces within strings
    showtabs=false,                 % show tabs within strings adding particular underscores
    frame=single,                   % adds a frame around the code
    rulecolor=\color{black},        % if not set, the frame-color may be changed on line-breaks within not-black text (e.g. commens (green here))
    tabsize=2,                      % sets default tabsize to 2 spaces
    captionpos=b,                   % sets the caption-position to bottom
    breaklines=true,                % sets automatic line breaking
    breakatwhitespace=false,        % sets if automatic breaks should only happen at whitespace
    title=\lstname,                   % show the filename of files included with \lstinputlisting;
                                    % also try caption instead of title
    keywordstyle=\color{blue},          % keyword style
    commentstyle=\color{dkgreen},       % comment style
    stringstyle=\color{mauve},         % string literal style
    escapeinside={\%*}{*},            % if you want to add LaTeX within your code
    morekeywords={*,...}               % if you want to add more keywords to the set
}
\setCJKmainfont{Noto Serif CJK TC} % 主要字体 Noto Serif
\definecolor{CVBlue}{RGB}{23,110,191}
\newfontfamily\Hack{Hack} % 代码字体

\begin{document}
% \pagenumbering{gobble} % suppress displaying page number

%%%% 利用tikz来定位照片,部分招聘单位可能需要“以貌取人”
% \begin{tikzpicture}[remember picture, overlay] 
%   \node[anchor = north east] at ($(current page.north east)+(-1cm,-1.2cm)$) {\includegraphics[height=2.5cm]{avatar}};
% \end{tikzpicture}%
%%%% 利用tikz来定位学校Logo,这里只在第一页显示,如果需要每页都有,可以考虑在页首、页尾或者background中加入,不过简历也就一两页,无所谓了
% \begin{tikzpicture}[remember picture, overlay] 
%   \node[anchor = north west] at ($(current page.north west)+(0.2cm,-0.2cm)$) {\includegraphics[height=2cm]{zte_logo}};
% \end{tikzpicture}%
%%%% 利用tikz来定位页尾栏,电子版简历使用,黑白纸质打印效果可能并不好。这里只在第一页显示,如果需要每页都有,页尾或者background中加入。
% \begin{tikzpicture}[remember picture, overlay] 
%   \node[anchor = south,fill=CVBlue,draw=none,minimum width=\paperwidth ,minimum height=1.5em ,align=center ,font=\footnotesize ,text=white] at ($(current page.south)$) {\faLinkedinSquare https://www.linkedin.com/in/username \qquad \faGithub https://github.com/HuangNO1 \qquad \faRssSquare http://huangno1.github.io/};
% \end{tikzpicture}%
%tikzpicture环境很敏感,注释周围的空格、空行都会引起水平距离或垂直距离的变化,
%

\centerline{\LARGE\bfseries{黄钰泯}}

\centerline{\normalsize{应聘岗位:Java开发工程师}}
\centerline{\normalsize{\faPhone\ (+86)180-6171-4729 \quad \faEnvelope\ \href{qqmail:1355104382@qq.com}{1355104382@qq.com}}}

% \section{\makebox[\faGraduationCap][c]{\color{CVBlue}\faGraduationCap}\  教育背景}

\section{\color{CVBlue}\faGraduationCap\  教育背景}

\textbf{南京邮电大学 \quad 中国南京 \quad 双一流}  \hfill 2017-09 $\sim$ 2021-06
\newline\quad 物联网学院 \quad 网络工程 \quad 本科 



\section{\color{CVBlue}\faBriefcase\ 工作经历}

\textbf{中兴通讯股份有限公司 \quad IT软件开发工程师} \hfill 2021-07 $\sim$ 至今

\textbf{主要职责与业绩}: \quad 在数字技术产品部-产品架构团队,担任后端开发工程师,参与过三个大项目的研发和维护。
在架构团队中作为主负责人参与架构治理相关工作(“架构评估模型”推广)。在团队中负责团队的代码设计评审,代码走查检查代码规范与逻辑,相关模块架构守护等工作。%在公司获得过黑客松大赛一等奖,部门最佳质量个人。

\section{\color{CVBlue}\faUsers\ 项目经历}

\textbf{iDTS数据同步服务 (SpringBoot、Redis、Mysql)} \hfill 2021-09 $\sim$ 2022-05

% 项目角色: \quad 项目开发人员
\begin{itemize}[parsep=0.5ex]
\item \textbf{主要职责与业绩}:后端开发,负责数据同步后端转换与微服务架构搭建,经过不断需求迭代,完成公司内部100+团队使用的数据同步服务落地,达成2000+数据同步任务里程碑。

\item \textbf{成长}:胜任复杂的开发需求,加深对数据库知识的掌握;数据同步方面,掌握状态机在定时同步中的理解与作用;
项目部署方面,了解系统的高可用多点部署,并在BCM断电演练中保证业务的正常运行,达到高可用2级以上标准;
开发方面,具备代码设计能力减少Bug产出,
并解决30+产品bug。在开发的过程中,逐渐熟悉完整的DevOps开发流程,并认识到团队管理中协同工作重要性。
\end{itemize}

% \textbf{架构度量平台 (Maven、Hbase)} \hfill 2022-06 $\sim$ 2022-07
% \begin{itemize}[parsep=0.5ex]
%   \item \textbf{主要职责与业绩}:从0到1开发上线架构度量平台,可视化微服务之间的调用关系,用于指导微服务的合并和拆分,通对pinpoint的数据进行分析处理,解析出1600+微服务之间的相互调用关系%使用大数据spark解决亿级别数据处理,该方案在部门方案大赛中获得第六名(6/64)
%   \item \textbf{成长}:熟悉了系统设计方案,从业务背景、关键需求、架构和技术选型等软件架构设计全流程,快速学习Hbase基础并上手完成相关功能的开发
%   \end{itemize}


\textbf{DN Market企业数字资产交易市场 (SpringBoot、Cola架构)} \hfill 2022-07 $\sim$ 2023-01

% 项目角色: \quad 项目开发人员
\begin{itemize}[parsep=0.5ex]
\item \textbf{主要职责与业绩}:依据敏捷开发模式快速迭代产品新需求,主要负责后台微服务中DDD的落地,以及前端部分功能的开发,完成50+需求开发与故障修复。

\item \textbf{成长}:从零开始参与此产品开发,理解多线程高并发服务设计、分布式锁保证事务、利用缓存达到性能提升,
注重代码规范性与边界值,并针对代码编写单元测试检查自己的代码逻辑正确性。 同时,使用Cola架构来组织后台的业务代码。
学习到数字化转型解决方案“数字星云”,了解到组装式架构的设计思想,并实际将这一思想在项目中执行落地。 %TODO 数字星云、组装式架构、DDD 三者联系 这个不好写
\end{itemize}

\textbf{中兴云云门户开发(SpringBoot、PostgreSQL、Golang)} \hfill 2023-02 $\sim$ 至今

% 项目角色: \quad 项目开发人员
\begin{itemize}[parsep=0.5ex]
\item \textbf{主要职责与业绩}:负责交易引擎四大服务和数据统计服务等后端开发。担任前期项目交接维护,
到后期跨部门协同开发,并负责用户运维工作,排查修复多环境bug。
\item \textbf{成长}:能胜任维护相关4个微服务的维护;熟悉容器化分布式系统在项目部署架构上的设计与实现;能对业务性能问题排查优化、
PostgreSQL索引优化,能通过索引提高查找效率,并系统性分析各环境下数据库的性能。
\end{itemize}

% \textbf{架构治理(架构指标)} \hfill 2022-02 $\sim$ 至今

% 项目角色: \quad 项目开发人员
% \begin{itemize}[parsep=0.5ex]
% \item \textbf{主要职责与业绩}:主要负责人对各项目架构治理,制定架构原则检查模型,架构模型检查,潜入研发流程
% \item \textbf{成长}:掌握架构治理相关指标和体系,熟悉架构落地的常见方式
% \end{itemize}


\section{\color{CVBlue}\faCogs\ 技术背景}
% increase linespacing [parsep=0.5ex]
\begin{itemize}[parsep=0.5ex]
  \item \textbf{程序语言}: $Java$(熟悉JUC并发编程、JVM虚拟机、熟悉GC机制、Java IO模型)
  \item \textbf{中间件和框架}:熟悉 Spring 框架、Mysql性能优化、微服务应用开发、Redis缓存、Kafka消息队列
  \item \textbf{DevOps}:了解CI/CD,熟悉敏捷研发流程
  \item \textbf{编码设计}:熟悉Java代码规范,数据库研发规范,设计模式,Cola架构
\end{itemize}

\section{\color{CVBlue}\faHeart\ 荣誉}

\textbf{中兴通讯部门 \quad 部门最佳质量个人} \hfill 2022-08

\textbf{中兴通讯部门黑客松大赛 \quad 团队第一名一等奖} \hfill 2023-09

\section{\color{CVBlue}\faChalkboard\ 自我评价}

\quad\quad 具备良好的分析问题和解决问题的能力,并编写微服务换包部署脚本提高团队80\%的工作效率。
保持着持续学习的习惯,可以快速应对新技术带来的挑战。在过去的工作
中,对于曾经解决的技术难点编写总结性文档,在团队内分享新技术,包含但不限Spring容器底层原理、Java源码中的设计模式等议题。
能识别团队技术风险,并主动牵头解决团队痛点问题。
\quad\quad 

\end{document}